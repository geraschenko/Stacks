\sektion{29}{Hilb and Quot}

 Recall that last time we saw that $\M_{1,1}$ was algebraic over $\ZZ[1/6]$. We did this
 by finding some open subset $U\subseteq \AA^2$ which was invariant under the $(4,6)$
 $\GG_m$ action on $\AA^2$. And we saw that $\M_{1,1}\cong [U/\GG_m]$. In general, you
 don't have such a delicate analysis, but this is the way to show algebraicity. You use
 the Hiblert scheme or the Quot functor.

 \underline{Aside on the Quot functor}. (Grothendieck, Seminars Boubaki, 1960/61, no.
 221) Let $f:X\to S$ be a separated morphism of finite presentation. Let $\L$ be
 relatively ample on $X$, and let $P\in \QQ[z]$ (the Hilbert polynomial). Also, fix a
 quasi-coherent sheaf $\F$ on $X$. Then we can define $\quot^P(\F/X/S):(\sch/S)^{op}\to
 \set$ by $(S'\to S)\mapsto \Bigl\{\F_{S'}\twoheadrightarrow \G| \F_{S'}$ the pullback of
 $\F$ to $X_{S'}=X\times_S S'$, $\G$ quasi-coherent and locally finitely presented
 (basically coherent), $\G$ has support proper over $S'$, and for every $s'\in S'$, the
 Hilbert polynomial (makes sense because may as well replace $X$ by the support\anton{I
 missed some stuff}) of $\G_{S'}=\G|_{X_{S'}}$ is equal to $P\Bigr\}/\cong$, where
 isomorphism is
 \[\xymatrix{
    \F_{S'}\ar@{->>}[r]\ar@{->>}[dr] & \G_1\ar[d]_\wr^f\\
    & \G_2
 }\]
 Such an iso is unique if it exists because of surjectivity.
 \begin{theorem}[Grothendieck]
   If $\F$ is locally finitely presented, $\quot^P(\F/X/S)$ is a quasi-projective
   $S$-scheme (projective if $X\to S$ is proper).
 \end{theorem}
 In fact, you get a nice projective embedding into a Grassmanian.
 \begin{remark}
   If $\F=\O_X$, then $\quot^P(\F/X/S)$ is usually written $\hilb^P_{X/S}$, the
   \emph{Hilbert scheme}.
 \end{remark}
 \begin{example}[Hartshorne's comment]
   Suppose $\F=\bigoplus \O_X$, then we can take $\G$ to be a quotient of any one of
   them, so it looks like we'll get an infinite disjoint union. Maybe the theorem only
   holds if $\F$ is locally finitely presented.
 \end{example}
 \begin{remark}
   If $f:X\to S$ is locally finitely presented and separated morphism of algebraic
   spaces, and $\F$ is quasi-coherent on $X$, then we can still define $\quot(\F/X/S)$.
   The ample part is only used for the Hilbert polynomial part. This is an algebraic
   space and is quasi-proper (satisfies the valuative criterion, but may not be finite
   something \anton{}) if $X\to S$ is proper.
 \end{remark}
 \begin{example}
   If $g\ge 2$, then $\M_g$ is algebraic. The idea is the same as for $\M_{1,1}$. Put
   some extra structure to get something representable by a scheme and then quotient out
   the extra structure. If $(\pi:C\to S)\in \M_g(S)$, then one can show (by Riemann-Roch)
   \begin{itemize}
     \item $(\Om^1_{C/S})^{\otimes 3}$ is relatively very ample.
     \item $V_C=\pi_*(\Om_{C/S}^{1})^{\otimes 3}$ has rank $5(g-1)$.
     \item $\pi_*(\Om^1_{C/S})^{\otimes n}$ has rank $(2n-1)(g-1)$ for $n\ge 3$.
   \end{itemize}
   This stuff implies that for all $s\in S$, the Hilbert polynomial of $C_s$ is
   $P=(6z-1)(g-1)$ (here $\L=(\Om^1_{C/S})^{\otimes 3}$) because $n=3z$. You have to
   compute the dimension of $H^0(\L^{\otimes m})= (6m-1)(g-1)$.

   Let $\tilde \M_g$ be the functor $\sch^{op}\to \set$ given by $S\mapsto
   \{(C/S,\sigma:\PP^{5g-6)}\xrightarrow\sim \PP V_C)\}/\cong$, where
   $(C_1,\sigma_1)\cong (C_2,\sigma_2)$ if there is an isomorphism $i:C_1\xrightarrow\sim
   C_2$ over $S$ such that
   \[\xymatrix{
    \PP^{5g-6}\ar[r]^{\sigma_1}\ar[dr]_{\sigma_2} & \PP V_{C_1}\ar[d]^{i} & C_1 \ar@{_(->}[l]\ar[d]^i\\
    & \PP V_{C_2} & C_2 \ar@{_(->}[l]
   }\]
   There is at most one such isomorphism because some stuff has to respect embeddings.
   Concretely, they are equivalent if the images of the curves in $\PP^{5g-6}$ are
   equal.

   Thus, we see that we get a subfunctor $j:\tilde \M_g\hookrightarrow
   \hilb^{(6z-1)(g-1)}_{\PP^{5g-6}}$. Let $P=(6z-1)(g-1)$.
   \begin{lemma}
     Let $X\hookrightarrow \PP^{5g-6}$ (a morphism over $S$) be an $S$-valued point of
     $\hilb^P_{\PP^{5g-6}}$. Then the set of points $s\in S$ for which $X_s$ is a smooth
     genus $g$ curve is an open set. This lemma is true without anything about $5g-6$;
     genus is constant in a flat family.
   \end{lemma}
   \begin{proof}
     (1) the condition that $X\to S$ is smooth is open in $S$. After some reduction, you
     can assume $S$ and $X$ are noetherian, so the set of points where stuff is not
     smooth is closed, and $X\to S$ is proper, so the image in $S$ is closed.

     (2) the semi-continuity theorem implies that the set of points where the fiber is
     geometrically connected is open.

     Now after shrinking on $S$, we can assume $X\to S$ is smooth proper with
     fibers geometrically connected.

     (3) $X_s$ smooth geometrically connected and the Hilbert polynomial is $(6z-1)(g-1)$
     implies that $X_s$ is a genus $g$ curve.
   \end{proof}
   \anton{Hartshorne: you need that some embedding is given by the tri-canonical system,
   so it looks like $\tilde \M_g$ is a closed subset of this open thing}
   \begin{corollary}
     $\tilde \M_g$ is represented by an open subscheme of $\hilb^P_{\PP^{5g-6}}$.
   \end{corollary}
   \textbf{To be corrected}

   Then we want to say that $\M_g\cong [\tilde\M_g/PGL_{5g-6}]$ (quotient out by the
   choice of the isomorphism $\sigma$). We have a map $\tilde \M_g\to \M_g$, given by
   $(C,\sigma)\mapsto C$. By uniqueness of isomorphisms, this gives a morphism of fibered
   categories. This induces a morphism of fibered categories
   \[
   [\tilde
   \M_g/PGL_{5g-6}]^{ps}\to \M_g.\tag{$\ast$}
   \]
   \begin{lemma}
     This map induces an isomorphism $[\tilde \M_g/PGL_{5g-6}]\xrightarrow\sim \M_g$.
   \end{lemma}
   \begin{proof}
     It is enough to show that $(\ast)$ is fully faithful, because it is clear that every
     point is locally in the image (we can always locally choose a basis for the locally
     free sheaf). Equivalently, given $(C_1,\sigma_1)$ and $(C_2,\sigma_2)$ in $\tilde
     \M_g(S)$ and an isomorphism $\iota:C_1\to C_2$, there is a unique $h\in PGL_{5g-6}(S)$
     such that
     \[\xymatrix{
      \PP^{5g-6} \ar[d]_h \ar[r]^{\sigma_1} & \PP V_{C_1} \ar[d]_\wr^\iota\\
      \PP^{5g-6} \ar[r]^{\sigma_2} & \PP V_{C_2}
     }\]
     and we see that there is no choice for $h$, it has to be the composition.
   \end{proof}
 \end{example}

 \begin{definition}
   An algebraic stack $\X$ over $S$ is \emph{Deligne-Mumford} if there exists an \'etale
   surjection $U\to \X$ with $U$ a scheme. (sometimes people require the diagonal to be
   finite).
 \end{definition}
 \begin{remark}
   $\M_{1,1}$ and $\M_g$ are Deligne-Mumford. The way we've done things, it looks like it
   might be hard to show this. We'll see that an algebraic stack $\X$ is Deligne-Mumford
   if and only if ``objects have no infinitesimal automorphisms''. It is a purely
   deformation-theoretic thing. For the $\M_g$ and $\M_{1,1}$ cases, this will follow
   from the fact that $H^0(C,T)=0$.
 \end{remark}

 Corrected proof that $\M_g$ is algebraic.

 It is still correct that $\M_g$ is represented by a locally closed subscheme of the
 Hilbert scheme, but this is hard (it requires knowledge of the relative Picard
 functor)\anton{there are some diagrams about this in Ed's notes, but I don't understand
 them}. The following argument will bypass this.

 Let $S$ be a fixed base scheme. Let $X$ be quasi-projective, flat, and finitely
 presented over $S$, and let $Y$ be quasi-projective and finitely presented over $S$. We
 define the functor $\uhom(X,Y):(\sch/S)^{op}\to \set$ by $(S'\to S)\mapsto
 \hom_{S'}(X_{S'},Y_{S'})$.
 \begin{proposition}
   $\uhom(X,Y)$ is a scheme.
 \end{proposition}
 \begin{proof}
   To give an element of $\hom_{S'}(X_{S'},Y_{S'})$, it is enough to give its graph (a
    subscheme of $X_{S'}\times_{S'}Y_{S'}=(X\times_S Y)_{S'}$), so we get a morphism
   $\uhom(X,Y)\to \hilb_{X\times_S Y}$, given by $(f:X_{S'}\to Y_{S'})\mapsto
   \Ga_f\subseteq (X\times_S Y)_{S'}$. Note that $\Ga_f$ is proper and flat over the base
   since $X$ is projective and flat \anton{}, so $\Ga_f$ is a point of the Hilbert
   scheme. Also, $Y$ is \anton{something}

   Claim: This identifies $\uhom(X,Y)$ with an open subscheme of $\hilb_{X\times_S Y}$.
   To see this, stare that the following diagram.
   \[\xymatrix{
    \Ga\ar@{^(->}[r]\ar[dr] & (X\times_S Y)\times_S \hilb_{X\times_S Y}\ar[r]\ar[d] & X\times_S
    \hilb_{X\times_S Y}\ar[dl]\\
    & \hilb_{X\times_S Y}
   }\]
   \anton{I haven't thought about this yet.}
 \begin{lemma}
   Let $f:\Ga\to X$ be a morphism of proper, flat, finitely presented $S$-schemes. Then
   there is an open set $U\subseteq S$ so that a morphism $S'\to S$ factors through $U$
   if and only if $\Ga_{S'}\to X_{S'}$ is an isomorphism.
 \end{lemma}
 \begin{proof}
   There is a fiber-wise criterion for being \'etale \cite[IV.17.8.2]{EGA}, which implies
   that if $\gamma\in \Ga$ is a point with image $s\in S$, then $\Ga_S\to X_S$ is \'etale
   at $\gamma$ if and only $\Ga\to X$ is \'etale at $\gamma$.

   Let $Z\subseteq \Ga$ be the closed subset where $f$ is not \'etale, and replace $S$ by
   the complement of the (closed) image of $Z$ in $S$ ($\Ga$ is proper over the base).
   This reduces to the case where $f$ is proper and \'etale (and therefore finite
   \'etale). Such a morphism has a rank, which we want to be 1. Let $W\subseteq X$ be the
   open and closed \anton{} subset where $rk(f_*\O_\Ga)>1$, and take $U$ to be the
   complement of the image of $W$ in $S$.
 \renewcommand{\qedsymbol}{$\square_\text{Lemma}$}
 \end{proof}
 \end{proof}

 Define $\M_g'$ to be the category of pairs $(C,\tau)$, where $\pi:C\to S$ is a genus $g$ curve over $S$ and $\tau:\O_S^{5(g-1)}\xrightarrow\sim V_C=\pi_*(\Om^1_{C/S})^{\otimes 3}$ is an isomorphism; the morphisms are what you expect. Define a morphism $\M_g'\to \hilb_{\PP^{5g-6}}$ by taking $(C,\tau)$ to the closed subscheme of $\PP_S^{5g-6}$ corresponding to the closed immersion $j_\tau:C\xhookrightarrow{\text{can}} \PP V_C\xrightarrow[\sim]{\PP\tau} \PP_S^{5g-6}$, or $\bigl(C,(\Om^1_{C/S})^{\otimes 3})\hookrightarrow \bigl(\PP V_C,\O_{\PP V_C}(1)\bigr)\xrightarrow\tau \bigl(\PP_S^{5g-6},\O(1)\bigr)$ \anton{I don't think this is clearer}
 
 Given $\pi:C\to S$, what is $\{\tau:\O_S^{5g-5}\xrightarrow\sim V_C\}$? It is close to $\{(j,\iota)|j:C\hookrightarrow \PP^{5g-6},\iota:j^\star\O(1)\xrightarrow\sim (\Om^1_{C/S})^{\otimes 3}\}$. In fact, there is a map $\tau\mapsto (j_\tau,\text{can})$. You can go in the other direction: given $(j,\iota)$, you get a map by the adjunction
 \[\xymatrix{
   \O_X^{5g-5}\ar[r]\ar@{.>}[d] & \pi_*j^*\O_{\PP^{5g-6}}(1)\ar[d]^\iota\\
   V_C\ar@{}[r]|<>(.5){\mbox{$=$}} & \pi_*(\Om^1_{C/S})^{\otimes 3}
 }\]
 Let $T$ be the total space of the $\GG_m$-torsor of isomorphisms, and let $\Phi$ be the fiber product shown ($\Phi$ classifies $(C/S,j,\iota)$)
 \[\xymatrix{
  \Phi\ar[r]\ar[d]\ar@{}[dr]|(.25)\pb & \uhom(C_U,T)\ar[d]\\
  U\ar[r]^\id & \uhom(C_U,C_U)
 }\qquad\qquad \xymatrix{
  T\ar[r] & C_U \ar@{^(->}[r]^{j_U}\ar[d] & \PP^{5g-6}_U\ar[dl]\\
  \Phi\ar[r] & U\ar@{}[r]|<>(.5){\mbox{$\subseteq$}} & \hilb_{\PP^{5g-6}}
 }\]
 More precisely, $\Phi$ is the functor $S\mapsto \{(C/S,j:C\hookrightarrow \PP^{5g-6},\iota:j^\star\O(1)\xrightarrow\sim (\Om^1_{C/S})^{\otimes 3})\}$ and $U$ is the funct $S\mapsto (C/S,j)$.

 To get $\M_g$: $\M_g'$ has an action of $GL_{5g-5}$. Say $g\in GL_{5g-5}$, then we define $g\cdot (C/S,\tau:\O^{5g-5}_S\xrightarrow\sim V_C)=(C/S,\tau\cdot g:\O_S^{5g-5}\xrightarrow\sim V_C)$
