\sektion{1}{Motivation: non-representable functors}

You should do homework (especially if you're enrolled). You'll have to do a lot of work, even if you already know a lot. We'll try to organize a discussion section. The prerequisite is schemes. The references are good; you should look at them. Vistoli's notes on Grothendieck topologies are good; Knutson's book is good, so we'll try to put it on reserve in the library.

\bigskip
In a (Nov.~5 1959) letter from Grothendieck to Serre, Grothendieck talks about moduli spaces and says that he keeps running up against the problem that objects have automorphisms.

A main point for today is that many interesting functors are not representable. If $X$ is a scheme, then we get a functor $h_X:\sch^{op}\to \set$ given by $Y\mapsto \hom(Y,X)$.
\begin{lemma}[Yoneda]
  The functor $h_-:\sch\to \fun(\sch^{op},\set)$ is fully faithful.
\end{lemma}
\begin{definition}
  A functor $F:\sch^{op}\to \set$ is \emph{representable} if $F\simeq h_X$ for some $X$.
  When you represent a functor $F$, you give the scheme $X$ together with the natural
  isomorphism $F\simeq h_X$. When you do this, $X$ is unique up to unique isomorphism.
\end{definition}
\begin{example}
  $\AA^n:Y\mapsto \Ga(Y,\O_Y)^n$. On morphisms, $\AA^n$ takes $g:Y'\to Y$ to the
  pull-back map $g^*:\Ga(Y,\O_Y)^n\to \Ga(Y',\O_{Y'})^n$. Let $X=\spec
  \ZZ[x_1,\dots, x_n]$. Then $\hom(Y,X)\cong \hom_{alg}\bigl(\ZZ[x_1,\dots,
  x_n],\Ga(Y,\O_Y)\bigr)\cong \Ga(Y,\O_Y)^n = \AA^n(Y)$ is a natural isomorphism, so
  $X$ represents $\AA^n$.
\end{example}
\begin{example}
  $\AA^n\smallsetminus \{0\}$ should be a sub-functor of $\AA^n$. We define it by
  $Y\mapsto \bigl\{ (y_1,\dots, y_n)\in \Ga(Y,\O_Y)^n|$for every $p\in Y$, the images
  of the $y_i$ are not all zero in $k(p)\bigr\}$. In the homework, you will show that
  this functor is representable.
\end{example}
\begin{definition}
  An \emph{elliptic curve} over a scheme $Y$ is a diagram
  $\xymatrix{E\ar[r]_f & Y\ar@/_/[l]_e}$ where $e$ is a section of $f$ and the
  fibers of $f$ are genus 1 curves.
\end{definition}
\begin{example}
  Let $\M_{1,1}$ be the functor defined by $Y\mapsto \{$isoclasses of elliptic curves
  over $Y\}$. If $g:Y'\to Y$ is a morphism of schemes, then we define
  $\M_{1,1}(g):\M_{1,1}(Y) \to \M_{1,1}(Y')$ to be the usual pull-back $g^*$.
  \[\xymatrix{
   \llap{$\M_{1,1}(g)(E)=\;$} g^* E \ar[r] \ar[d]_{g^*f} \ar@{}[dr]|(.25){\pb} & E\ar[d]^f\\
   Y'\ar[r]^g & Y
  }\qquad\qquad
  \xymatrix{
   g^* E \ar[r] & E\\
   Y' \ar@(l,d)_(.3)\id \ar[r]_g  \ar@{-->}[u]^{\exists!} \ar[ur]_{e\circ g}
   & Y \ar@/_/[u]_e
  }\]
  The section $Y'\to g^*E$ is induced by $\id_Y$ and $e\circ g$ by the universal
  property of pull-backs.
\end{example}
\begin{proposition}\label{lec01:M11notrep}
  $\M_{1,1}$ is not representable.
\end{proposition}
The intuitive reason that $\M_{1,1}$ is not representable is the following lemma. It
says that you cannot have any kind of twisting of bundles.
\begin{lemma}
  If $F$ is a representable functor, with $s_1,s_2\in F(Y)$ and a covering $Y=\bigcup
  U_i$ such that $s_1|_{U_i}=s_2|_{U_i}$ for all $i$, then $s_1=s_2$.
\end{lemma}
\begin{proof}
  We have that $F\cong h_X$ for some $X$, and $s_1$ and $s_2$ are given by morphisms $Y\to
  X$ that they agree on a cover of $Y$. Since morphisms glue, $s_1=s_2$.
\end{proof}
Unfortunately, to show that $\M_{1,1}$ is not representable via this Lemma, you need to
generalize your notion of covering (to \'etale covers). We'll see these later. For now
we'll give another proof.
\begin{proof}[Proof of \ref{lec01:M11notrep}]
  Assume there is a scheme $M$ and an isomorphism $\M_{1,1}\simeq h_M$. Let $k$ be an
  algebraically closed field of characteristic not 2. Consider
  $R=k[\lambda]_{\lambda(1-\lambda)}$, so $\spec R$ is $\AA^1_k$ with 0 and 1 removed.
  Let $E\subseteq \PP^2_R$ be the closed subscheme defined by $y^2z=x(x-z)(x-\lambda
  z)$, so the fibers $E_\lambda$ of the natural map $E\to \spec R$ are genus 1 curves.
  We define $e:\spec R\to D(z)\cong \spec R[x,y]$ by the map $R[x,y]\to R$, $x,y\mapsto
  0$ and observe that the image of $e$ lies in $E$, and $e$ is a section of $E\to \spec
  R$.

  Observe that there is an action of $S_3$ on $R$ generated by $\sigma_0:\lambda\mapsto
  1/\lambda$ and $\sigma_1:\lambda\mapsto 1/(1-\lambda)$. Let
  $j=2^8 \frac{(\lambda^2-\lambda+1)^3}{\lambda^2(\lambda-1)^2}\in k(\lambda)$.
  \begin{lemma}
    The fixed points $k(\lambda)^{S_3}$ are exactly the elements of $k(j)$.
  \end{lemma}
  \begin{proof}
    First check that $j\in k(\lambda)^{S_3}$. Then we have that $k(j)\subseteq
    k(\lambda)^{S_3}\subseteq k(\lambda)$. By Galois theory, the second extension is
    degree 6, and the total extension is degree 6, so the first two fields are equal.
  \renewcommand{\qedsymbol}{$\Box_\text{Lemma}$}
  \end{proof}
  There are two steps remaining in the proof.\\
  (1) Let $E_\eta$ be the generic fiber of $E\to \spec R$. It is given by a map $\phi:\spec
  k(\lambda)\to M$. We claim that this map has to factor through the obvious map
  $g:\spec k(\lambda)\to \spec k(j)$.
  \[\xymatrix@C-1pc{
    & M \\
   \llap{$\eta =\:$}\spec k(\lambda) \ar[rr]^g \ar[ur]^\phi & & \spec k(j) \ar@{-->}[ul]_{\psi}
  }\qquad\qquad \qquad
  \xymatrix{
   \llap{$\phi \leftrightsquigarrow$\,} E_\eta \ar[r] \ar@{}[dr]|(.25)\pb \ar[d] & \tilde E \rlap{$\:\leftrightsquigarrow \psi$} \ar[d]\\
   \spec k(\lambda) \ar[r]^g &\spec k(j)
  }\]
   As we see from the diagram on the left, a factorization of $\phi$ is the same thing
  as a morphism $\hom(g,M):\phi\mapsto \psi$. By the natural isomorphism $h_M\simeq
  \M_{1,1}$, this is the same as a morphism $\M_{1,1}(g): \tilde E\mapsto E$, where $\tilde
  E$ is the elliptic curve over $k(j)$ defined by $\psi$, as shown in the diagram on the
  right.

  Thus, we have $E_\eta = \tilde E\times_{\spec k(j)} \spec k(\lambda)$. This implies
  that the action of $S_3$ can be lifted to $E_\eta$ by acting on the second factor.\\
  (2) Then we check that the $S_3$ action cannot lift to $E_\eta$.

  (1) Say $x=\phi(\eta)$, then we have
  \[\xymatrix{
  \prod_{\sigma:k(\lambda)\hookrightarrow \bbar{k(j)}} \bbar{k(j)} & k(\lambda) \ar[l]_<>(.5){\prod \sigma} & k(x)\ar[l] \\
  \bbar{k(j)} \ar@{^(->}[u]_{\text{diagonal}}& k(j) \ar@{^(->}[u] \ar[l] &
  }\]
%   \[\xymatrix{
%   k(x)\ar[r] & k(\lambda) \ar[r]^<>(.5){\prod \sigma} & \prod_{\sigma:k(\lambda)\hookrightarrow \bbar{k(j)}} \bbar{k(j)}\\
%   & k(j) \ar@{}[u]|{\large \cup} \ar[r] & \bbar{k(j)} \ar@{^(->}[u]_{\text{diagonal}}
%   }\]
   where $\sigma$ runs over all embeddings of $k(\lambda)$ into $\bbar{k(j)}$ over
  $k(j)$ (there are six such embeddings). It is enough to show that
  \[\xymatrix{
   \coprod_{\sigma:k(\lambda)\hookrightarrow \bbar{k(j)}} \spec\bbar{k(j)} \ar[r]^<>(.5)f \ar[d] & M\\
   \spec \bbar{k(j)} \ar@{-->}[ur]
  }\]
   \anton{why is this factorization enough?} Fact: If $E_1,E_2$ have the same
  $j$-invariant, then they are isomorphic.

  But for any $\sigma:k(\lambda)\hookrightarrow \bbar{k(j)}$ the $j$-invariant of
  $E_\eta \times_{\spec k(\lambda), \sigma} \spec \bbar{k(j)}$ is $j$. \anton{$j$ is a
  regular function on $\spec R$. The $j$-invariant of $E_\eta$ is $j\in
  k(\eta)=k(\lambda)$. What does $j$-invariant mean for curves not over points in $\spec
  R$?} So when $f$ is restricted to any particular $\spec \bbar{k(j)}$, it is always the
  same map, so $f$ factors through $\spec \bbar{k(j)}$.

  (2) recall that $E_\eta\to \spec k(\lambda)$ is the subset of $\PP^2_{k(\lambda)}$
  defined by $y^2z=x(x-z)(x-\lambda z)$. Elliptic curves have a group structure and
  torsion points. In fact, $E_\eta[2]$, the 2-torsion points, are $\infty=[0,1,0],
  [0,0,1], [1,0,1]$ and $[\lambda,0,1]$. If $S_3$ acts on $E_\eta$, then it must
  preserve the 2-torsion points (and it must fix $\infty$, which is the identity
  element). How could we lift $\sigma_0:\lambda\mapsto 1/\lambda$? It would have to lift
  \[\xymatrix @R-1pc{
   E_\eta \ar[r]^{\tilde \sigma_0}  \ar[dd] & E_\eta \ar[dd]\\ \\
   \spec k(\lambda) \ar[r] & \spec k(\lambda)\\
   1/\lambda & \lambda \ar@{|->}[l]
  }\]
   So we need an endomorphism of $k(\lambda)[x,y]/\bigl(y^2-x(x-1)(x-\lambda)\bigr)$
  which inverts $\lambda$. Working case-by-case, we can show that the only hope you have
  must be of the form $x\mapsto \lambda x$ and $y\mapsto uy$ for some unit $u$ (this all
  comes from the fact that we must preserve 2-torsion points). From that we get
  $u^2y^2=\lambda^3 x(x-1)(x-\lambda)$ \anton{I don't get that}, so $u^2=\lambda^3$, so
  $\lambda^3$ has a square root in $k(\lambda)$. Since $\lambda$ is an indeterminant,
  this is a contradiction.
\end{proof}
$\spec R$ represents (essentially) $Y\mapsto \bigl(E/Y$ an elliptic curve, with a basis
for its 2-torsion$\bigr)$ (it actually represents this together with some $\omega\in
f_*\Omega'_{E/Y}$).

Consider the functor $G:\sch^{op}\to \set$ given by $Y\mapsto $ isomorphisms classes of
pairs $\bigl(E/Y, \O_Y^3\xrightarrow{\sim} f_* \O_E(3e)\bigr)$, where $f:E\to Y$.
\begin{proposition}
  $G$ is representable.
\end{proposition}
\begin{proof}
  \[\xymatrix{
                   & h_Z \ar[d] \ar@{}[r]|- {\mbox{$\subseteq$}} & h_{\PP^2\times \hilb}\\
   G\ar[r] \ar[ur] & h_{\hilb  (\PP^2)} }\] This gives that $G$ is represented by an
  open sub-scheme of $Z$. \anton{I can't make sense of this}
\end{proof}
