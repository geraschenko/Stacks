\documentclass[12pt,twoside]{article}
 
 \usepackage{amsmath}       % I think this gives me some symbols
 \usepackage{amsthm}        % Does theorem stuff
 \usepackage{amssymb}       % more symbols and fonts
 \usepackage{empheq}        % Some more extensible arrows, like \xmapsto
% \usepackage{mathabx}
 \usepackage{mathrsfs}      % Sheafy font \mathscr{}
% \usepackage{picinpar}      % for pictures in paragraphs
 \usepackage{pstricks}      % PSTricks!
% \usepackage[active]{srcltx}
% \usepackage{xcolor}        % for colors (links)
 \usepackage[all]{xy}       % Include XY-pic
    \SelectTips{cm}{10}     % Use the nicer arrowheads
    \everyxy={<2.5em,0em>:} % Sets the scale I like
 \usepackage[colorlinks,
             linkcolor=black,
             pagebackref,
             bookmarksnumbered=true]{hyperref}

%%%%%%% Pagestyle stuff %%%%%%%%%%%%%%%%%%%
 \usepackage{fancyhdr}                   %%
   \pagestyle{fancy}                     %%
   \fancyhf{} %delete the current section for header and footer
 \usepackage[paperheight=11in,%          %%
             paperwidth=8.5in,%          %%
             outer=1.2in,%               %%
             inner=1.2in,%               %%
             bottom=.7in,%               %%
             top=.7in,%                  %%
             includeheadfoot]{geometry}  %%
   \addtolength{\headwidth}{.75in}       %%
   \fancyhead[RO,LE]{\thepage}           %%
   \fancyhead[RE,LO]{\sectionname}       %%
   \setlength{\headheight}{15.8pt}       %%
   \raggedbottom                         %%
%%%%%%% End Pagestyle stuff %%%%%%%%%%%%%%%


%%%%%%% Stuff for keeping track of sections %%%%%%%%%%%%%%%%%%%%%%%%%%%%
 \newcommand{\sektion}[2]{\stepcounter{section} \renewcommand{\thesection}{#1} \newpage\section{#2} \gdef\sectionname{#1\quad #2}}
 \newcommand{\subsektion}[1]{\subsection*{#1} \addcontentsline{toc}{subsection}{#1}}       %%
 %This is the empty section title, before any section title is set    %%
 \newcommand\sectionname{}                                            %%
%%%%%%% End stuff for keeping track of sections %%%%%%%%%%%%%%%%%%%%%%%%

% %%%%%%% Stuff for keeping track of sections %%%%%%%%%%%%%%%%%%%%%%%%%%%%
%  \newcommand{\sektion}[1]{\newpage\section{#1}%                       %%
%                           \gdef\sectionname{\thesection\quad #1}}     %%
%  \newcommand{\subsektion}[1]{\subsection*{#1}%                        %%
%                          \addcontentsline{toc}{subsection}{#1}}       %%
%  %This is the empty section title, before any section title is set    %%
%  \newcommand\sectionname{}                                            %%
% %%%%%%% End stuff for keeping track of sections %%%%%%%%%%%%%%%%%%%%%%%%

%%%%%%%%%%%%%%% Theorem Styles and Counters %%%%%%%%%%%%%%%%%%%%%%%%%%
 \renewcommand{\theequation}{\thesection.\arabic{equation}}         %%
 \makeatletter                                                      %%
    \@addtoreset{equation}{section} % Make the equation counter reset each section
    \@addtoreset{footnote}{section} % Make the footnote counter reset each section
                                                                    %%
 \newenvironment{warning}[1][]{%                                    %%
    \begin{trivlist} \item[] \noindent%                             %%
    \begingroup\hangindent=2pc\hangafter=-2                         %%
    \clubpenalty=10000%                                             %%
    \hbox to0pt{\hskip-\hangindent\manfntsymbol{127}\hfill}\ignorespaces%
    \refstepcounter{equation}\textbf{Warning~\theequation}%         %%
    \@ifnotempty{#1}{\the\thm@notefont \ (#1)}\textbf{.}            %%
    \let\p@@r=\par \def\p@r{\p@@r \hangindent=0pc} \let\par=\p@r}%  %%
    {\hspace*{\fill}$\lrcorner$\endgraf\endgroup\end{trivlist}}     %%
                                                                    %%
 \newenvironment{exercise}[1][]{\begin{trivlist}%                   %%
    \item{\bf Exercise\@ifnotempty{#1}{ #1}.}\it}{\end{trivlist}}   %%
 \newenvironment{solution}{\begin{trivlist}%                        %%
    \item{\it Solution.}}{\end{trivlist}}                           %%
                                                                    %%
 \def\newprooflikeenvironment#1#2#3#4{%                             %%
      \newenvironment{#1}[1][]{%                                    %%
          \refstepcounter{equation}                                 %%
          \begin{proof}[{\rm\csname#4\endcsname{#2~\theequation}%   %%
          \@ifnotempty{##1}{\the\thm@notefont \ (##1)}\csname#4\endcsname{.}}]%%
          \def\qedsymbol{#3}}%                                      %%
         {\end{proof}}}                                             %%
 \makeatother                                                       %%
                                                                    %%
 \newprooflikeenvironment{definition}{Definition}{$\diamond$}{textbf}%
 \newprooflikeenvironment{example}{Example}{$\diamond$}{textbf}     %%
 \newprooflikeenvironment{remark}{Remark}{$\diamond$}{textbf}       %%
                                                                    %%
 \theoremstyle{plain}                                               %%
 \newtheorem{theorem}[equation]{Theorem}                            %%
 \newtheorem*{claim}{Claim}                                         %%
 \newtheorem*{lemma*}{Lemma}                                        %%
 \newtheorem*{theorem*}{Theorem}                                    %%
 \newtheorem{lemma}[equation]{Lemma}                                %%
 \newtheorem{corollary}[equation]{Corollary}                        %%
 \newtheorem{proposition}[equation]{Proposition}                    %%
%%%%%%%%%%% End Theorem Styles and Counters %%%%%%%%%%%%%%%%%%%%%%%%%%

%%% These three lines load and resize a caligraphic font %%%%%%%%%
%%% which I use whenever I want lowercase \mathcal %%%%%%%%%%%%%%%
 \DeclareFontFamily{OT1}{pzc}{}                                 %%
 \DeclareFontShape{OT1}{pzc}{m}{it}{<-> s * [1.100] pzcmi7t}{}  %%
 \DeclareMathAlphabet{\mathpzc}{OT1}{pzc}{m}{it}                %%
                                                                %%
%%% and this is manfnt; used to produce the warning symbol %%%%%%%
 \DeclareFontFamily{U}{manual}{}                                %%
 \DeclareFontShape{U}{manual}{m}{n}{ <->  manfnt }{}            %%
 \newcommand{\manfntsymbol}[1]{%                                %%
    {\fontencoding{U}\fontfamily{manual}\selectfont\symbol{#1}}}%%
%%%%%%%%%%%%%%%%%%%%%%%%%%%%%%%%%%%%%%%%%%%%%%%%%%%%%%%%%%%%%%%%%%

%%%%%%%%%%%%%%%% Anton's Macros %%%%%%%%%%%%%%%%%%%%%%%%%%%%%%%%%%%%%%
 \newcommand{\anton}[1]{[[\ensuremath{\bigstar\bigstar\bigstar} #1]]}
 \renewcommand{\AA}{\mathbb{A}}
 \newcommand\A{\mathcal{A}}
 \newcommand{\ab}{\text{\sf Ab}}
 \newcommand{\algsp}{\text{\sf AlgSp}}
 \newcommand{\aff}{\text{\sf Aff}}
 \DeclareMathOperator{\ann}{ann}
 \DeclareMathOperator{\aut}{\underline{Aut}}
 \DeclareMathOperator{\Aut}{Aut}
 \newcommand{\B}{\mathcal{B}}
% \newcommand{\bbar}[1]{\overline{#1}}
 \newcommand{\bbar}[1]{\setbox0=\hbox{$#1$}\dimen0=.2\ht0 \kern\dimen0 \overline{\kern-\dimen0 #1}}
 \newcommand{\CC}{\mathbb{C}}
 \newcommand{\C}{\text{\sf C}}
 \newcommand{\cat}{\text{\sf Cat}}
 \newcommand{\catfont}[1]{\text{\sf #1}}
 \newcommand{\coh}{\text{\sf Coh}}
 \DeclareMathOperator{\coker}{coker}
 \newcommand{\D}{\text{\sf D}}
 \newcommand{\e}{\varepsilon}
 \DeclareMathOperator{\End}{\ensuremath{\mathcal{E}\kern-.125em\mathpzc{nd}}}
 \newcommand{\E}{\mathcal{E}}
 \newcommand{\F}{\mathcal{F}}
 \newcommand{\FF}{\mathbb{F}}
 \DeclareMathOperator{\fun}{\sf Fun}
 \newcommand{\G}{\mathcal{G}}
 \newcommand{\Ga}{\Gamma}
 \newcommand{\GG}{\mathbb{G}}
 \newcommand{\gpoid}{\text{\sf Gpoid}}
 \renewcommand{\H}{\mathcal{H}} % old \H{x} is an x with a weird umlaut in text mode
 \DeclareMathOperator{\hilb}{Hilb}
 \let\hom\relax % kills the old hom, which is lowercase
 \DeclareMathOperator{\hom}{Hom}
 \DeclareMathOperator{\Hom}{\mathcal{H}\kern-.125em\mathpzc{om}}
 \DeclareMathOperator{\HOM}{HOM}
 \renewcommand{\labelitemi}{--}  % changes the default bullet in itemize
 \newcommand{\I}{\mathcal{I}}
 \DeclareMathOperator{\id}{id}
 \DeclareMathOperator{\im}{im}
 \DeclareMathOperator{\isom}{\underline{Isom}}
 \newcommand{\J}{\mathcal{J}}
 \newcommand{\K}{\mathcal{K}}
 \renewcommand{\L}{\mathcal{L}}
 \newcommand{\liset}{\text{\it lis-et}}
 \newcommand{\Liset}{\text{\it Lis-Et}}
 \newcommand{\m}{\mathfrak{m}}
 \newcommand{\M}{\mathcal{M}}
 \renewcommand\mod{\textrm{-\sf mod}}
 \newcommand{\N}{\mathcal{N}}
 \DeclareMathOperator{\nat}{Nat}
 \DeclareMathOperator{\Nat}{NAT}
 \renewcommand{\O}{\mathcal{O}}
 \newcommand{\Om}{\Omega}
 \newcommand{\p}{\mathfrak{p}}
 \newcommand{\pb}{\rule{.4pt}{5.4pt}\rule[5pt]{5pt}{.4pt}\llap{$\cdot$\hspace{1pt}}}
 \newcommand{\po}{\rule{5pt}{.4pt}\rule{.4pt}{5.4pt}\llap{$\cdot$\hspace{1pt}}}
 \renewcommand{\P}{\mathcal{P}}
 \newcommand{\PP}{\mathbb{P}}
 \DeclareMathOperator{\proj}{Proj}
 \DeclareMathOperator{\Proj}{\mathcal{P}\kern-.125em\mathpzc{roj}}
 \newcommand{\qco}{\text{\sf Qcoh}}
 \newcommand{\QQ}{\mathbb{Q}}
 \DeclareMathOperator{\quot}{Quot}
 \renewcommand{\r}{\mathfrak{r}} % old \r{x} is a little circle over x in text mode
 \newcommand{\RR}{\mathbb{R}}
 \newcommand{\sch}{\text{\sf Sch}}
 \newcommand{\set}{\text{\sf Set}}
 \renewcommand{\setminus}{\smallsetminus}
 \newcommand{\sh}{\text{\sf Sh}}
 \DeclareMathOperator{\spec}{Spec}
 \DeclareMathOperator{\Spec}{\mathcal{S}\!\mathpzc{pec}}
 \DeclareMathOperator{\supp}{Supp}
 \newcommand{\T}{\text{\sf T}}
 \newcommand{\topoi}{\text{\sf Topoi}}
 \newcommand{\tors}{\text{\sf Tors}}
 \DeclareMathOperator{\tot}{Tot}
 \DeclareMathOperator{\Tot}{\ensuremath{\mathpzc{Tot}}}
 \newcommand{\ttilde}[1]{\widetilde{#1}}
 \newcommand{\udot}{\ensuremath{{\lower .183333em \hbox{\LARGE \kern -.05em$\cdot$}}}}
 \newcommand{\uudot}{{\ensuremath{\!\mbox{\large $\cdot$}\!}}}
 \DeclareMathOperator{\uhom}{\underline{Hom}}
 \newcommand{\U}{\mathcal{U}}
 \newcommand{\V}{\mathcal{V}}
 \newcommand\VV{\mathbb{V}}
 \newcommand{\w}{\omega}
 \newcommand{\W}{\mathcal{W}}
 \newcommand{\X}{\mathcal{X}}
 \newcommand{\Y}{\mathcal{Y}}
 \newcommand{\Z}{\mathcal{Z}}
 \newcommand{\ZZ}{\mathbb{Z}}
%%%%%%%%%%%% End Anton's Macros %%%%%%%%%%%%%%%%%%%%%%%%%%%%%%%%%%%%%%%

 \openout0=lastupdated.html
 \write0{\today}
 \closeout0
