\sektion{E}{Exercises and solutions}
 
\subsektion{Exercise set 1}
\begin{exercise}[1.1]
 If you have never done so, prove Yoneda's lemma: The association $X\to h_X$ defines a fully faithful functor from $\C$ to the category of functors $\C\to \set$.
\end{exercise}
\begin{solution}
 Before we prove the theorem, let's state it slightly more generality.
 \begin{theorem*}[Yoneda's Lemma]
  For any functor $F:\C^\circ\to \set$, there is a natural bijection ${\rm Nat}(h_X,F)\cong F(X)$. In particular, taking $F=h_Y$, we see that the functor $h_{-}:\C\to \fun(\C^\circ,\set)$ is a fully faithful embedding of categories.
 \end{theorem*}
 \begin{proof}
  Given $\eta\in \nat(h_X,F)$, we have $\eta(X):\hom(X,X)\to F(X)$, so we get an element $a=\eta(X)(\id_X)\in F(X)$. Conversely, given $a\in F(X)$, we define a natural transformation $\eta$ by taking $f\in h_X(Y)=\hom(Y,X)$ to $\eta(Y)(f):=(Ff)(a)$. Check that these are inverses, and that the bijection is natural in $F$ and $X$. The following diagram should help:
  \[\raisebox{5.5pc}{$\xymatrix @C=10pc @R=4pc{
  \id_X \ar@{|->}[d] \ar@{}[dr]|(0.06){
     \xymatrix @!0 @R=3pc @C=6.5pc{
     \hom(X,X)\ar[d]_{- \circ f} & F(X) \ar[d]^{Ff}\\
     \hom(Y,X) \ar[r]^(.6){\eta(Y)} & F(Y)}
  } & a\ar@{|->}[d]\\
  f \ar@{|->}[r] & (Ff)(a)
  }$}\qedhere\]
  \anton{maybe this should be done in two steps: (1) We say a functor $G\colon \C^\circ\to \set$ has a \emph{universal point} $y\in G(Y)$ for some object $Y$ if for any $x\in G(X)$, there is a unique morphism $f\colon X\to Y$ such that $Gf(y)=x$. If $y\in G(Y)$ is a universal point, then for any functor $F$, $\nat(G,F)\cong F(Y)$. (2) $h_X$ has the universal point $\id_X\in h_X(X)$}
 \end{proof}
\end{solution}

\begin{exercise}[1.2]
  (a) Let $n\ge 1$ be an integer and let $GL_n:\sch^\circ\to \set$ be the functor sending
  a scheme $Y$ to the set $GL_n\bigl(\Ga(Y,\O_Y)\bigr)$. Prove that $GL_n$ is a
  representable functor.\\
  (b) Let $X$ represent $GL_n$. Prove that the group structure on
  $GL_n\bigl(\Ga(Y,\O_Y)\bigr)$ induces morphisms
  \[
     m:X\times X\to X,\quad i:X\to X,\quad e:\spec \ZZ\to X
  \]
  such that the following diagrams commute:
  \[\hspace*{-1em}
   \xymatrix{\spec \ZZ \times X \ar[r]^<>(.5){e\times \id} \ar@{=}[dr] & X\times X \ar[d]^m \\ & X}
   \quad
   \xymatrix{X\times \spec \ZZ \ar[r]^<>(.5){\id\times e} \ar@{=}[dr] & X\times X \ar[d]^m \\ & X}
   \quad
   \xymatrix{
    X\times X\times X \ar[r]^<>(.5){m\times \id} \ar[d]_{\id\times m} & X\times X\ar[d]^m\\
    X\times X \ar[r]^<>(.5)m & X
   }\]
  \[
   \xymatrix{X\ar[r]^<>(.5){i\times \id}\ar[d] & X\times X \ar[d]^m\\ \spec \ZZ \ar[r]^<>(.5)e & X}
   \qquad\qquad
   \xymatrix{X\ar[r]^<>(.5){\id\times i}\ar[d] & X\times X \ar[d]^m\\ \spec \ZZ \ar[r]^<>(.5)e & X}
  \]
\end{exercise}
\begin{solution}
  (a) Let $Z=\spec \bigl(\ZZ[x_{11},\dots, x_{nn},y]/(1-y \det X)\bigr)$, where $X$ is the
  matrix with $(i,j)$-th entry $x_{ij}$. The claim is that $Z$ represents $GL_n$. To see
  this, note that
  \[\hom_{\sch} (Y,Z) \cong \hom_\catfont{Ring}\Bigl(\frac{\ZZ[x_{11},\dots,
  x_{nn},y]}{(1-y \det X)}, \Ga(Y,\O_Y)\Bigr) \cong GL_n\bigl(\Ga(Y,\O_Y)\bigr).\]

  (b)  The functor $GL_n(-)$ factors through $\catfont{Gp}$, so $GL_n(-)\simeq\hom(-,X)$ is a group object in the category $\fun(\sch^\circ,\set)$, i.e.~it has maps like $m$, $i$, and $e$, satisfying the diagrams above, with $\spec \ZZ$ replaced by the final object, $\hom(-,\spec \ZZ)$. Since the Yoneda embedding is a fully faithful, we have that $X$ is a group object in $\sch$, as desired.
\end{solution}

\begin{exercise}[1.3]
  (a) Let $\AA^n\setminus \{0\}:\sch^\circ\to \set$ be the functor sending
  a scheme $Y$ to the set of $n$-tuples $(y_1,\dots, y_n)$ of sections $y_i\in
  \Ga(Y,\O_Y)$ such that for every point $y\in Y$ the images of the $y_i$ in $k(y)$ are
  not all zero. Show that $\AA^n\setminus \{0\}$ is representable.\\
  (b) Let $(\AA^n\setminus \{0\})/\GG_m:\sch^\circ\to \set$ be the functor
  sending a scheme $Y$ to the quotient of the set $(\AA^n\setminus\{0\})(Y)$ by the
  equivalence relation $(y_1,\dots, y_n)\sim (y_1',\dots, y_n')$ if there exists a unit
  $u\in \Ga(Y,\O_Y^\times)$ such that $y_j=uy_j'$ for all $j$. Show that $(\AA^n\setminus
  \{0\})/\GG_m$ is not representable.
\end{exercise}
\begin{solution}
  (a) Something like ``an open subfunctor of a representable functor is representable''.
  You can check explicitly that this is represented by the open subscheme of $\AA^n$
  obtained by removing the closure of the point $(x_1,\dots, x_n)$ (note that this point
  isn't closed!).

  (b) Call the functor in question $F$. A representable functor is a sheaf on the
  Zariski site because morphisms glue and morphisms which agree locally agree globally.
  We will show that $F$ is not representable by showing it is not a sheaf.

  Let $k$ be a field, and let $U=\spec k[x]$ and $V=\spec k[1/x]$ be the usual open sets
  in $\PP^1_k$. We have that $\Ga(\PP^1_k,\O_{\PP^1_k})=k$, so
  $F(\PP^1_k)=(k^n\setminus\{0\})/k^\times$. We have sections $[x:1:\cdots:1]\in F(U)$
  and $[1:1/x:\cdots:1/x]\in F(V)$ which are not restrictions of global sections on
  $\PP^1_k$ (because every global section can be represented by an $n$-tuple in $k$).
  However, on the intersection $\spec k[x,1/x]$, the two sections agree. Therefore, $F$
  is not a sheaf, so it is not representable.

  \smallskip
  Ishai's solution (sketch): $\AA^n\smallsetminus \{0\}$ is the functor $Y\mapsto
  \{\O_Y^n\twoheadrightarrow \mathcal{L}, \varphi:\mathcal{L}\cong \O_Y\}$. The functor
  $F$ is $Y\mapsto \{\O_Y^n\twoheadrightarrow \mathcal{L},\text{ with }\mathcal{L}\cong
  \O_Y,\text{ but you don't care how}\}$. Since there are sheaves which are locally
  trivial but not globally trivial, this functor is not a sheaf. The sheafification is
  represented by $\PP^{n-1}$, and it is $Y\mapsto \{\O_Y^n\twoheadrightarrow
  \mathcal{L},\text{ with } \mathcal{L}\text{ invertible}\}$ \dots this approach is
  discussed in Hartshorne's section on morphisms to projective space.
\end{solution}

\begin{exercise}[1.4]
  Let $\catfont{Top}$ be the category of topological spaces with morphisms being continuous maps. Let $F:\catfont{Top}^\circ\to\set$ be the functor sending a topological space $S$ to the collection $F(S)$ of all its open sets.\\
  (a) Endow $\{0,1\}$ with the coarsest topology in which the subset $\{1\}\subset \{0,1\}$ is closed. Show that the open sets in this topology are $\varnothing$, $\{0\}$, and $\{0,1\}$.\\
  (b) Show that $\{0,1\}$ with the above topology represents $F$.\\
  (c) Let $\catfont{HausTop}\subseteq \catfont{Top}$ denote the full subcategory of Hausdorff topological spaces. Show that the restriction $F|_\catfont{HausTop}:\catfont{HausTop}^\circ\to \set$ is not representable.
\end{exercise}
\begin{solution}
  (a) $\varnothing$ and $\{0,1\}$ will be open in any topology, and $\{0\}$ must be open for $\{1\}$ to be closed, so these three sets must be open. On the other hand, these three sets form a topology, so this is the coarsest such topology.

  (b) Given any continuous map $f:S\to \{0,1\}$, we get an open set $f^{-1}(0)\subseteq S$. Conversely, given any open set $U\subseteq S$, the function $f_U$ given by $f_U(U)=0$ and $f_U(S\setminus U)=1$ is continuous, so $F(S)\cong\hom(S,\{0,1\})$. It wasn't specified what $F$ did on morphisms, but the most obvious thing is to pull back open sets along morphisms, as is done by $\hom(-,\{0,1\})$.

  (c) Assume $F_\catfont{HausTop}$ is represented by a Hausdorff space $X$. Since the 1-point Hausdorff space $\ast$ has two open sets, $|\hom(\ast,X)|=|X|=2$. Since $X$ is Hausdorff, it must be a 2-point set with the discrete topology. Let $I$ be the unit interval, then there are only two elements of $\hom(I,X)$ since $I$ is connected, but I has an uncountable number of open sets, a contradiction. Thus, $F_\catfont{HausTop}$ is not representable.
\end{solution}

\begin{exercise}[1.5]
  Another proof that $\M_{1,1}$ is not representable.\\
  (a) Let $k$ be a field and $k\hookrightarrow \bbar k$ an algebraic closure. Show that
  if $F$ is a representable functor then the map $F(\spec k)\to F(\spec \bar k)$ is
  injective.\\
  (b) Let $D> 1$ be an integer. Show that the two elliptic curves over $\QQ$
  \[
     E_1: y^2=x^3+x,\quad E_2:y^2=x^3+Dx
  \]
  are isomorphic over $\overline \QQ$, but not over $\QQ$. From this and (a), conclude that
  the functor $\M_{1,1}$ defined in class is not representable.
\end{exercise}
\begin{solution}
  (a) Let $f:\spec \bar k\to \spec k$ be the obvious map. Assume $F\simeq \hom(-,X)$ and
  that $g,h:\spec k\to X$ are two morphisms such that $g\circ f=h\circ f$, i.e.~two
  elements of $F(\spec k)$ that map to the same element of $F(\spec \bar k)$. Let $\spec
  R$ be an open affine neighborhood of the image of $g\circ f=h\circ f$. Then the images
  of $g$ and $h$ lie in $\spec R$, and we have that $f^\sharp\circ g^\sharp =
  f^\sharp\circ h^\sharp:R\to k\xhookrightarrow{f^\sharp}\bar k$. Since $f^\sharp$ is an
  injection, we must have that $g^\sharp=h^\sharp$, so $g=h$. Thus, $F(\spec k)\to
  F(\spec \bar k)$ is injective.

  (b) If $D$ is a fourth power, then you can do some change of coordinate operations to show
  that $E_1\cong E_2$. Since everything is a fourth power in $\overline \QQ$, the two are
  isomorphic over $\overline \QQ$.

  Something shows that they are not isomorphic over $\QQ$, like looking at torsion points,
  perhaps after reducing at some prime dividing $D$.
\end{solution}


\subsektion{Exercise set 2}
\begin{exercise}[2.1]
  For a set $S$, let $\Aut(S)$ denote the group of bijections $S\to S$. An \emph{action}
  of a group $G$ on $S$ is defined to be a homomorphism $G\to \Aut(S)$. Let $G$ be a
  group and define $BG$ to be the category with one object $*$ and with
  $\hom(*,*)=G$.\\
  (a) Show that if $F$ is a presheaf on $BG$ then $F(*)$ admits a natural action of $G$.\\
  (b) Show that the induced functor $($presheaves on $BG)\to ($sets with $G$-action$)$,
  given by $F\mapsto F(\ast)$, is an equivalence of categories.
\end{exercise}

\begin{exercise}[2.2]
  (a)\\
  (b)\\
  (c)\\
  (d)
\end{exercise}

\begin{exercise}[2.3]
  (a)\\
  (b)\\
  (c)\\
  (d)
\end{exercise}

\begin{exercise}[2.4]
  Let $\C$ be a site with associated topos $\T$, and let $X\in \C$ be an object. Assume
  that the functor of points $h_X$ is a sheaf.\\
  (a) Show that the topology on $\C$ induces a topology on $\C/X$.\\
  (b) Show that the category of sheaves on $\C/X$ is equivalent to $\T/h_X$. In
  particular, $\T/h_X$ is a topos.\\
  (c) Define $j^*:\T\to \T/h_X$ by sending $F$ to $F\times h_X$ with the projection to
  $h_X$. Show that $j^*$ commutes with finite projective limits and has a right adjoint
  $j_*$, given by $(j_*G)(Y)=\hom_{\T/h_X}(h_Y^a\times h_X,G)$. In particular, there is
  a morphism of topoi $j:\T/h_X\to \T$.
\end{exercise}
\begin{solution}
  (a) Let $\C$ be a site, let $\D$ be category, and let $F:\D\to \C$ be a functor which
  respects fiber products, then we can declare $\{Y_i\to Y\}$ to be a covering of an
  object $Y\in \D$ if $\{FY_i\to FY\}$ is a covering of the object $FY\in \C$. It is
  immediate to verify that this satisfies the axioms of a Grothendieck topology. In our
  case, we take $F$ to be the forgetful functor $\C/X\to \C$.

  (b) Given a sheaf $F:\C^\circ\to \set$, with a morphism of sheaves $\eta:F\to h_X$, we
  can define a sheaf $\tilde F$ on $\C/X$ by $(f:U\to X)\mapsto \eta(U)^{-1}(f)$. To
  make sense of this, note that $\eta(U):F(U)\to h_X(U)$, and $f\in h_X(U)$, so $\tilde
  F(f:U\to X)$ is a subset of $F(U)$. One can then check that $\tilde F$ is a presheaf
  (functor) and satisfies the sheaf axiom.

  Conversely, if $G:(\C/X)^\circ\to \set$ is a sheaf on $\C/X$, then we can define
  a sheaf $G':\C\to \set$ by $U\mapsto \coprod_{f\in h_X(U)}G(f:U\to X)$, with the
  natural projection to $h_X(U)$. Again, one checks that $G'$ is a sheaf and $G'\to h_X$
  is a morphism of sheaves.

  I've swept it under the rug, but you can define what these two operations do to
  morphisms of sheaves (it is the obvious thing), and they are inverse (up to
  isomorphism), so we get the desired equivalence of categories.

  (c) Note that for an object
  $(G\to h_X)\in \T/h_X$ and a sheaf $F\in \T$, we have that $\hom_{T/h_X}(G\to
  h_X,j^*F)=\hom_{\T}(G,F)$. That is, $j^*$ is right adjoint to the forgetful functor
  $\C/X\to \C$, so it commutes with all projective limits.

  \anton{prove that $j_*$ is right adjoint to $j^*$}
\end{solution}

\subsektion{Exercise set 3}
\begin{exercise}[3.1]
  For a scheme $S$ let $\M_0(S)$ denote the category whose objects are smooth proper
  morphisms $C\to S$ all of whose geometric fibers are smooth connected curves of genus
  0. For an fppf and quasi-compact cover $X\to Y$ define the category $\M_0(X\to Y)$ as
  in class. Show that the pullback functor $\M_0(Y)\to \M_0(X\to Y)$ is an equivalence
  of categories (hint: consider the dual of the canonical sheaf).
\end{exercise}
\begin{solution}
  The dual of the canonical bundle is a functorial choice of relatively very ample
  sheaf. Then the proof of descent for $\M_g$ ($g\ge 2$) works word for word.
\end{solution}

\begin{exercise}[3.2]
  Let $f:X\to Y$ be a faithfully flat morphism of locally noetherian schemes.\\
  (a) Show that a quasi-coherent sheaf $F$ on $Y$ is coherent if and only if $f^\star F$
  on $X$ is coherent.\\
  (b) Show that a quasi-coherent sheaf $F$ on $Y$ is locally free of finite rank if and
  only if $f^\star F$ is locally free of finite rank.\\
  (c) Prove that the pullback functor $\qco(Y)\to \qco(X\to Y)$ identifies the category
  of coherent (resp.~locally free of finite rank) sheaves on $Y$ with the fully
  subcategory of $\qco(X\to Y)$ consisting of pairs $(E_X,\sigma)$, where $E_X$ is
  coherent (resp.~locally free of finite rank).
\end{exercise}
\begin{solution}
  Being coherent (resp.~locally free of finite rank) is local, so we may take open affines
  $V=\spec A\subseteq Y$ and $U=\spec B\subseteq f^{-1}(V)\subseteq X$.

  If $F$ is coherent (resp.~locally free of finite rank), then it is $\tilde M$ for some
  $A$-module $M$ which is finitely generated (resp.~free of finite rank). Then it is clear
  that $f^\star F=(M\otimes_A B)^\sim$ is coherent (resp.~locally free of finite rank).

  If $F=\tilde M$ is quasi-coherent on $V$, and $f^\star F=(M\otimes_A B)^\sim$ is coherent
  (resp.~locally free of finite rank) on $U$. Then $M\otimes_A B$ is finitely generated
  (resp.~free of finite rank). Let $\{\sum_j m_{ij}\otimes b_{ij}\}_i$ be a finite
  (resp.~finite free) set of generators for $M\otimes_A B$. It is enough to check locally
  that $M$ is finitely generated \anton{is this true? No! Take the $k[x]$-module
  $M=\bigoplus_{a\in k} k[x]/(x-a)$ for some $k=\bar k$} (resp.~free of finite rank
  (Ex.~II.5.7)). Let $\mathfrak p \in \spec A$ and let $\mathfrak P\in \spec B$ lie over
  $\mathfrak p$. Then $B_\mathfrak P$ is faithfully flat over $A_\mathfrak p$ \anton{is
  this true?}.Thus, we may assume that $B$ is faithfully flat over $A$.

  (coherence) The map $A^N\to M$, given by $1_{ij}\mapsto m_{ij}$ is
  surjective after tensoring with $B$, so it is surjective, so $M$ is finitely
  generated over $A$.

  (loc.~free)

  (c)  Let $\P$ be any property of quasi-coherent sheaves so that for any fppf cover
  $f$, $F$ has $\P$ if and only if $f^\star F$ does.

  A $\P$ sheaf on $Y$ pulls back to some $E_X$ which is $\P$ on $X$. Conversely, assume
  $E_X$ is $\P$. By the equivalence of categories, it is $f^\star F$ for some $F$ on
  $Y$. Then $F$ (the image of $E_X$ under the equivalence) is also $\P$.
\end{solution}

\begin{exercise}[3.3]
  For a scheme $Y$ let $\aff(Y)$ denote the category whose objects are affine
  $Y$-schemes and whose morphisms are $Y$-morphisms. For a morphism of schemes $X\to Y$
  let $\aff(X\to Y)$ denote the category whose objects are pairs $(Z_X,\sigma)$, where
  $Z_X\in \aff(X)$ and
  \[
     \sigma:Z_X\times_{X,\textrm{pr}_1} (X\times_Y X)\to
     (X\times_Y X)\times_{\textrm{pr}_2,X} Z_X
  \]
  is an isomorphism of $X\times_Y X$-schemes satisfying a cocycle condition on
  $X\times_Y X\times_Y X$.\\
  (a) Write out precisely the cocycle condition in this case.\\
  (b) Prove that if $X\to Y$ is a quasi-compact fppf cover then the pullback functor
  $\aff(Y)\to \aff(X\to Y)$ is an equivalence of categories.
\end{exercise}
\begin{solution}
  (a) To save ink, we use the usual notation: write $Z_X\times_{X,\textrm{pr}_1}(X\times X)$
  as $p_1^* Z_X$, and similarly define $p_2^* Z_X$. Then $\sigma$ is an isomorphism
  $p_1^*Z_X\to p_2^*Z_X$. The cocycle condition is that following diagram must commute.
  \[\xymatrix @R-1pc @C-1pc {
   &
   p_{13}^*p_1^*Z_X \ar[rr]^{p_{13}^*\sigma} \ar@{=}[dl] & &
   p_{13}^*p_2^*Z_X \ar@{=}[dr]
   &
   \\
   p_{12}^*p_1^*Z_X \ar[dr]_{p_{12}^*\sigma} & & & &
   p_{23}^*p_2^*Z_X \\
   &
   p_{12}^*p_2^*Z_X \ar@{=}[rr] & &
   p_{23}^*p_1^*Z_X \ar[ur]_{p_{23}^*\sigma}\\
  }\]

  (b)  An affine $X$-scheme is $\Spec\A$ for some quasi-coherent sheaf of
  $\O_X$-algebras $\A$. Use descent for quasi-coherent sheaves of algebras.
\end{solution}

\begin{itemize}
\item[(4)] \textit{Let $B$ be a base scheme, $d\ge 1$ an integer, and fix a projective
$B$-scheme $P$ with an ample line bundle $L$. For a $B$-scheme $Y$, let $\M_g(P,d)(Y)$
denote the category whose objects are pairs $(C,f)$, where $C\to Y$ is an object of
$\M_g(Y)$ and $f:C\to P$ is a morphism such that the restriction of $f^*L$ to each
geometric fiber of $C\to Y$ is an ample sheaf of degree $d$. A morphism $(C',f')\to
(C,f)$ is a morphism $\varepsilon:C'\to C$ in $\M_g(Y)$ (so an isomorphism $C'\to C$ over
$Y$) such that $f\circ \varepsilon=f'$.}

\item [(4a)] \textit{For a morphism of schemes $X\to Y$ define the category
$\M_g(P,d)(X\to Y)$.}

Objects in $\M_g(P,d)(X\to Y)$ are pairs $\bigl((C,f),\sigma\bigr)$, where $(C,f)\in
\M_g(P,d)(X)$ and $\sigma$ is an isomorphism $p_2^*(C,f)\xrightarrow\sim p_1^*(C,f)$
satisfying the usual cocycle condition. A morphism $\bigl((C,f),\sigma\bigr)\to
\bigl((C',f'),\sigma'\bigr)$ is a morphism $\varepsilon:(C,f)\to (C',f')$ such that the
following diagram commutes.
\[\xymatrix{
   p_2^*(C,f) \ar[d]_\sigma \ar[r]^{p_2^*\varepsilon} & p_2^*(C',f') \ar[d]^{\sigma'}\\
   p_1^*(C,f) \ar[r]^{p_1^*\varepsilon} & p_1^*(C',f')
}\]

\item[(4b)] \textit{Prove that for an fppf quasi-compact cover $X\to Y$ pullback defines
an equivalence $\M_g(P,d)(Y)\to\M_g(P,d)(X\to Y)$.}

If $f^*L$ some fixed tensor power of it had to be relatively \emph{very} ample, then the
proof of descent for $\M_g$ would carry over.

\end{itemize}

\subsektion{Exercise set 4}

\subsektion{Exercise set 5}

\subsektion{Exercise set 6}

\subsektion{Exercise set 7}
