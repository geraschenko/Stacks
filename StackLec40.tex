
\subsektion{Proof: III}

Recall that we have $\X$ an algebraic stack of finite type with
finite diagonal over an noetherian affine base $S$. The goal is to
get a coarse space $\X\to X$ with a bunch of properties. So far,
we've done this if $\X$ admits a finite flat surjection from a
quasi-projective scheme.

We need three theorems.

\begin{theorem}[A, Zariski's main theorem for stacks?]
  Assume $\X$ admits a quasi-finite flat surjection $U\to \X$ with
$U$ a quasi-projective $S$-scheme. Then there is an algebraic stack
$\W$ over $S$ and a surjective separated \'etale morphism $\pi:\W\to
\X$ which is representable by schemes (for $T$ a scheme, $\W\times_\X
T$ is a scheme) and a closed immersion $Z\subseteq U\times_\X\W$ such
that $Z\to \W$ is a finite flat surjection and such that
  \begin{enumerate}
   \item for every quasi-compact open substack $\W'\subseteq \W$, the
preimage $Z'\subseteq Z$ is quasi-projective over $S$, and
   \item for every algebraically closed field $k$ and $w\in \W(k)$,
then $\aut_{\W(k)}(w)\to \aut_{\X(k)}(\pi(w))$ is an isomorphism (it
is always an injection by representability of $\pi$).
  \end{enumerate}
\end{theorem}
If $\X$ is quasi-compact, we get that $\W$ has a coarse moduli space
$W$, and $\W\times_\X \W$ will also have a coarse moduli space $R$,
and we'd like to show that $R\rightrightarrows W$ is an \'etale
equivalence relation. The second condition is what will force $R$ to
be \'etale.
\begin{theorem}[B]
  Let $\Y'\to \Y$ be a separated representable \'etale and
quasi-compact morphism of algebraic stacks of finite type over $S$
with finite diagonals. Assume there is a finite flat covering $U\to
\Y$ with $U/S$ quasi-projective and for every algebraically closed
field $k$ and $y'\in \Y'(k)$ mapping to $y\in \Y(k)$, the map
$\aut_{\Y'(k)}(y')\to \aut_{\Y(k)}(y)$ is an isomorphism. Then $\Y'$
admits a finite flat covering $U'\to \Y'$ with $U'/S$
quasi-projective and the map on coarse spaces $Y'\to Y$ is \'etale.
\end{theorem}
This, together with the first theorem will say that $\W\times_\X \W$
has a coarse space and the projections $R\rightrightarrows W$ are
\'etale.
\begin{theorem}[C]
  Let $\X/S$ be an algebraic stack locally of finite type with finite
diagonal. Then there is an open covering $\X=\bigcup \X_i$ such that
each $\X_i$ admits a quasi-finite flat surjection $U_i\to \X_i$ with
$U_i/S$ quasi-projective. (If $\X$ is DM, then you could replace
``flat'' by \'etale.)
\end{theorem}
Now we start with a general $\X$. Apply Theorem C to reduce to the
case where you have a quasi-finite flat surjection. Then we get
$R\rightrightarrows W$ an \'etale equivalence relation and we win.

Let's start with theorem A.
\begin{proof}[Proof of Theorem A]
  Let $\H$ (for Hilbert) be the stack over $S$ whose objects are
triples $(T,x,Z)$ such that $T$ is an $S$-scheme, $x\in \X(T)$, and
$Z\subseteq T\times_{x,\X} U$ is a closed subscheme (because it is
quasi-finite, this is a quasi-finite separated $T$-scheme, so it is
quasi-affine) flat over $T$. There is a natural forgetful map $\H\to
\X$ given by $(T,x,Z)\mapsto (T,x)$ and we have that
  \[\xymatrix{
    \hilb_{T\times_\X U} \ar@{}[dr]|(.25)\pb \ar[r]\ar[d] & \H\ar[d]\\
    T \ar[r] & \X
  }\]
  This implies that $\H$ is an algebraic stack with finite diagonal.
Maybe a better notation for $\H$ would be $\hilb_{U/\X}$.

  Define $\W'\subseteq \H$ to be the maximal open substack where
$\H\to \X$ is \'etale. In general if you have a finite type morphism
of schemes, the locus where the morphism is \'etale is an open set.
You can do the same thing here.
  \[\xymatrix{
     & \H_T\ar[d] \ar[r] & \H\ar[d]\\
    T'\ar@<.5ex>[r]\ar@<-.5ex>[r] & T\ar[r]^{sm} & \X
  }\]

  Define $I_\X$, the \emph{inertia stack}, to be
  \[\xymatrix{
   I_\X \ar[r]\ar[d] & \X\ar[d]^\Delta\\
   \X\ar[r]^\Delta & \X\times \X
  }\qquad\qquad
  \xymatrix{
   \aut_x \ar[r]\ar[d] & I_\X\ar[d]\\
   T\ar[r]^x & \X
  }\]
  An automorphism in $I_\X$ is a pair of automorphisms in $\X$. Now
define $I_\X'=I_\X\times_\X \W'$. Then we have that
  \[\xymatrix@C-2pc{
    I_{\W'} \ar[dr]\ar[rr]^j& & I_\X' \ar[dl]\\
    & \W'
  }\]
  \begin{lemma}
   $j$ is an open and closed immersion.
  \end{lemma}
  \begin{proof}
   $I_{\W'}$ and $I_\X'$ are both finite over $\W'$. The map $j$ is
also a monomorphism since $\W'\to \X$ is representable.
   \[\xymatrix{
     \aut_{(x,Z)}\ar[r] \ar[d] & \aut_x\ar[dl]\\
     T\ar[r]_{(x,Z)}& \W'
   }\]
   To prove the lemma, it is enough to show that $j$ is \'etale. For
this, look at the following picture.
   \[\xymatrix{
    I_{\W'}\ar[d]\ar[r]^{et} & I_\X' \ar[r]\ar[d] &
\W\ar[d]_\Delta\ar[dr]\\
    \W'\ar[r]^{et}\ar[dr]_{et} & \K \ar[r]\ar[d] & \W\times
\W\ar[d]_{et} & \X\ar[dl]^\Delta\\
    & \X\ar[r]^\Delta & \X\times \X
   }\]
  \end{proof}
  Aside: if $k=\bar k$, $x\in \X(k)$, $Z\subseteq U\times_{\X,x}k$,
then $\Aut(x,Z)$ is the set of automorphisms $\sigma:x\to x$ in
$\X(k)$ such that $\sigma:U\times_{\X,x}k\to U\times_{\X,x}k$ is an
automorphism over $Z$.

  Let $I_{\X}''\subseteq I_\X '$ be the complement of $I_{\W'}$, and
let $\W\subseteq \W'$ be the complement of the image of $I_\X''$ in
$\W'$. Here we use the fact that $I_\X'$ is proper over $\W'$ and so
$\I_\X''$ is also proper over $\W'$. The claim is that $\W$ works.
  \[\xymatrix{
   Z\ar@{^(->}[r] \ar[dr] & U\times_\X \W\ar[d]\\ & \W
  }\]
  We still need to show that $\W\to \X$ is surjective. Let $k$ be an
algebraically closed field and fix $x\in \X(k)$. Consider $w\in
\H(k)$ corresponding to
  \[\xymatrix{
    U\times_\X \spec k \ar[r]^\id \ar[dr] & U\times_{\X,x} \spec
k\ar[d]\\
    & \spec k
  }\]
  we have to show that something lies in the \'etale locus. Something
about the identity not working for a \emph{quasi}-finite morphism.
Let $V\to \X$ be a smooth surjection, and assume that $x$ comes from
a point $v:\spec k\to V$ (since $k=\bar k$). Then we have
  \[\xymatrix{
    & \hilb_{P/\hat V} \sqcup (else)\ar@{=}[d]\\
    w\in \hilb_{U\times_\X \spec k} \ar@{^(->}[r]\ar[d]&
\hilb_{U\times_\X V} \ar[r]\ar[d]& \H\ar[d]\\
    \spec k \ar[r]^v & V\ar[r] & \X
  }\]
  We can replaced $V$ by a completion (strict hensilization) at $v$.
Now look at $U\times_\X \hat V=P\sqcup Q$ where $P\to \hat V$ is
finite and $Q$ has empty closed fiber. Something with that top
$\hilb$ because of Hilbert polynomial considerations. Moreover, that
$\hilb$ is $\hat V\sqcup (rest)$, and our point lies in $\hat V$,
which proves the result.
\end{proof}
\begin{proof}[Outline proof of Theorem B]
  $\pi:\Y'\to \Y$ representable \'etale quasi-compact and $U\to \Y$
finite flat with $U/S$ quasi-projective, and for all $y'\in \Y'(k)$,
the map $\Aut(y')\to \Aut(\pi(y'))$ is an isomorphism.

  (a) $\Y'$ admits a finite flat surjection $U'\to \Y'$ with $U'/S$
quasi-projective. To see this,
  \[\xymatrix{
   U'\ar[r]\ar[d] & \Y'\ar[d]\\ U\ar[r] & \Y
  }\]
  $U'\to U$ \'etale quasi-compact so quasi-affine and quasi-affine
over quasi-projective ($U$) is quasi-projective. \anton{}

  (b) $Y'\to Y$ is \'etale. Idea of proof: you certainly have finite
type since we're in the noetherian case, so it is enough to check
formally \'etale. Now we need to write down what the complete local
rings of $Y'$ and $Y$ look like at a point. Fix $y\in Y(k)$ with $k$
algebraically closed, and fix a lifting $u\in U(k)$ of $y$, and let
$y'\in Y'(k)$ mapping to $y$. It takes a little argument, but you can
find $u'\in U'(k)$ mapping to $u$. Take $\hat \O_{Y,y}^{sh}$ (sh
means strict hensilization; else think closed point and completion at
local rings), it is $Eq(\hat \O^{sh}_{U,u}\rightrightarrows
\prod_{\xi\in R(k), s(\xi)=u=t(\xi)} \hat \O^{sh}_{R,\xi})$ with
$R=U\times_\Y U$. And we get $\hat \O_{Y,y'}^{sh} = Eq(\hat
\O^{sh}_{U',u'}\rightrightarrows \prod_{\xi'\in R'(k),
s(\xi')=u=t(\xi')} \hat \O^{sh}_{R',\xi'})$. The map between them is
induced by maps on the terms in the equalizers, the map $\hat
\O^{sh}_{U,y}\to \hat \O^{sh}_{U',y'}$ is an isomorphism and
something on the product terms. We want that $\{\xi'\in
R'(k)|s(\xi')=t(\xi')=u'\}\to \{\xi\in R(k)|s(\xi)=t(\xi)=u\}$ to be
bijective. The first set is in bijection with $\aut_{\Y'}(y')$ and
the second is in bijection with $\aut_{\Y}(y)$.
\end{proof}


