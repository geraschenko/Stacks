\sektion{28}{More about Algebraic Stacks; Examples}

 \begin{lemma}
   Let $f:\X\to X$ be a representable morphism from a stack to an
algebraic space. Then
   $\X$ is an algebraic space.
 \end{lemma}
 \begin{proof}
   First we check that $\X$ is equivalent to a sheaf (i.e.~that it is
fibered in discrete
   groupoids). To do this, it is enough to show that there are no
non-identity
   automorphisms in the fibers of $\X$. Let $x\in \X(T)$ for some
scheme $T$, thought of
   as a morphism $x:T\to \X$.

   (1) Let $F:(\sch/S)^{op}$ be the presheaf given by $(T\to
X)\mapsto \{$isomorphism
   classes in $\X(T)\}$. Then $\X\to F^{fib}$ is an equivalence. That
is, $\X(T)\to
   \pi_0\bigl(\X(T)\bigr)$ is an equivalence of categories
(\anton{exercise} it is enough
   to check that objects of $\X(T)$ have no non-trivial
automorphisms). To see this, let
   $x\in \X(T)$, then we have
   \[\xymatrix{
    (x,\id,\id)\ar@{}[r]|{\mbox{$\in$}} &\X\times_X T \ar[r]\ar[d]&
T\ar[d]^{f(x)}\\
    & \X \ar[r] & X
   }\]
    An automorphism of $x$ gives an automorphism of $(x,\id,\id)$,
which can't happen
    because the product is an algebraic space (so fibered in sets).

    (2) Check that $F$ is an algebraic space. It is a sheaf for free.

    (a) $F\to F\times_X F$ is representable (by
    schemes). Let $T\to F\times F$ be a morphism from a scheme, then
$T\times_{F\times F}
    F\cong (F\times_X T)\times_{(F\times_X T)\times_T (F\times_X T)}
T$, so it is a
    scheme (because $F\times_X T$ is an algebraic space, so its
diagonal is representable). This implies that $F\to F\times F$ is
representable \anton{as in appendix}

    (b) Let $U\to X$ be an \'etale cover by a scheme, let
$V=F\times_X U$ (which is
    an algebraic space), and let $V'\to V$ be an \'etale cover by a
scheme. Then $V'\to
    F$ is an \'etale cover.
 \end{proof}
 \begin{corollary}
   If $\X$ is an algebraic stack, then any morphism $x:X\to \X$ from
an algebraic space
   is representable.
 \end{corollary}
 \begin{proof}
   Let $T$ be a scheme with a morphism to $\X$.
   \[\xymatrix{
    P_U\ar[r]\ar[d] & U\ar[d]\\
    P\ar[r]\ar[d] & X\ar[d]\\
    T\ar[r] & \X
   }\]
    We see that $P\to X$ is representable (for any $U$, $P_U$ is an
algebraic space
   because it is $U\times_\X T$), so $P$ is an algebraic space by the
lemma.
 \end{proof}
 \begin{example}
   This is a main source of algebraic stacks. Let $Y/S$ be an
algebraic space, and let
   $G/S$ be a smooth group scheme. Then $\X=[Y/G]$ is an algebraic
stack. Recall that
   $[Y/G](T\to S)$ is the groupoid of diagrams $T\xleftarrow{G_T}
P\xrightarrow\rho Y$, where $P\to T$ is a
   $G$-torsor and $P\to Y$ is $G$-equivariant.
   \begin{proof}
     Representability of the diagonal: consider
     \[\xymatrix{
      \isom\bigl((P_1,\rho_1),(P_2,\rho_2)\bigr) \ar[r]\ar[d]
        & T\ar[d]^{(P_1,\rho_1)\times(P_2,\rho_2)}\\
      \X \ar[r]^<>(.5)\Delta & \X\times \X
     }\qquad\qquad
     \xymatrix{
      P_1\ar[dd]\ar[drr]^{\rho_1}\\
      & P_2\ar[r]_{\rho_2}\ar[dl] & Y\\
      T
     }\]
      $I:(\sch/T)^{op}\to \set$ is given by $(T'\to T)\mapsto
     \{\sigma:P_{1,T'}\xrightarrow\sim P_{2,T'}$ such that
$\rho_1=\rho_2\circ \sigma\}$.
     To show that $I$ is an algebraic space, it is enough to consider
the case where
     there exist sections $s_1:T\to P_i$ (because we can work \'etale
locally, and
     torsors are locally trivial).

     We see that $I\subseteq G_T$. $G_T$ is in bijection with
isomorphisms $\sigma:P_1\to
     P_2$, given by $\sigma\mapsto$ the unique $g$ such that
$\sigma(s_1)=gs_2$. We have
     \[\xymatrix{
      I\ar[d]\ar[r]\ar@{}[dr]|(.25)\pb & Y\ar[d]\\
      G_T \ar[d]\ar[r] & Y\times Y\\
      T
     }\qquad
      g\mapsto \bigl(\rho_1(s_1),g\rho_2(s_2)\bigr)
     \]
      (Aside: If $Y$ is separated, then $I$ is a closed subscheme of
$G_T$. If $G$ was
      affine, we'd get descent for schemes, so $I$ would be a scheme.)

      Smooth cover: $Y\to [Y/G]$.What does $Y$ represent? If $T\to
[Y/G]$, what does it
      mean for the map to factor through $Y$? It is the same as
choosing a section $T\to
      T\times_{[Y/G]}Y$. Thus, $Y$ represents the functor of triples
$(P,\rho,s)$, where
      $s$ is a trivialization of the torsor $P$.
   \end{proof}
 \end{example}
 \begin{example}
   If $Y=\spec k$ and $G$ is a smooth group scheme, then the
universal torsor is $\spec
   k\to [\spec k/G]$, which looks kind of boring, but thats what it
is.
 \end{example}
 \begin{example}[Weighted projective stack]
   Let $\alpha_0,\dots, \alpha_n\in \ZZ$. Then we define the
\emph{weighted projective stack} $\tilde \PP(\alpha_0,\dots,
\alpha_n) = [(\AA^{n+1}\setminus\{0\})/\GG_m]$, where the action is
given by $u\cdot (x_0,\dots, x_n)=(u^{\alpha_0}x_0,\dots,
u^{\alpha_n}x_n)$.
 \end{example}
 Application to $\M_{1,1}$. Recall that objects of $\M_{1,1}$ are
pairs
 $(S,\xymatrix@-1pc{E\ar[r] & S\ar@/_/[l]})$ and morphisms are
cartesian diagrams.
 \[\xymatrix{
    E'\ar[d]\ar[r] & E\ar[d]\\
    S'\ar[r]\ar@/^/[u] & S\ar@/_/[u]
 }\]
  Goal: see this in a weighted projective stack (in fact,
$\bbar\M_{1,1}\cong
 \tilde\PP(4,6)$ over $\ZZ[1/6]$).

 Let's start with $Y:(\sch/\ZZ[1/6])^{op}\to \set$, given by $S\mapsto
 \{(E/S,e,b:\O_S\xrightarrow\sim \w_{E/S})\}/\cong$, where
$\w_{E/S}=f_*\Om^1_{E/S}$ (this
 is locally free of rank 1 and commutes with arbitrary base change),
and where an
 isomorphism of elliptic curves $\sigma:E\xrightarrow\sim E'$ induces
an isomorphism
 $f_*(\sigma):\w_{E/S}\xrightarrow\sim \w_{E'/S}$, and we only allow
the isomorphisms
 which work well with $b$ and $b'$. In fact, we could make a
category, but it turns out
 that that category is equivalent to this set.

 As a stack, $\M_{1,1}$ should be $Y/\GG_m$ because $b$ is unique up
to the action of
 $\GG_m$.
 \begin{proposition}
   $Y$ is represented by $Y'=\spec
\ZZ[1/6][g_2,g_3][1/\Delta]\subseteq
   \AA^2_{\ZZ[1/6]}$, where $\Delta=g_2^3-27g_3^2$.
 \end{proposition}
 The action of $\GG_m$ is given by $g_2\mapsto u^4g_2$, $g_3\mapsto
u^6g_3$. We'll see
 that $\M_{1,1}=[Y/\GG_m]$.
 \begin{proof}
   (This is in \cite{Hartshorne}) Define $\xymatrix@-1pc{E'\ar[r] &
Y'\ar@/_/[l]_e}$ by
   $(2y)^2=4x^3-g_2x-g_3$, with $b'=-dx/2y$ giving us an isomorphism
   $\O_{Y'}\xrightarrow\sim f_*\Om^1_{E'/Y'}$. So this gives us a
morphism of functors
   $Y'\to Y$. To check that it is an isomorphism, we need to show
that for every scheme
   $S=\spec A$ (we can assume affine because we could define it as a
stack and check
   sheafy, so we can work locally) and $(E,e,b)\in Y(S)$, there
exists a unique $g_2,g_3$
   such that $(E,e,b)$ is given by $(E',e',b')_{(g_2,g_3)}$.

   We have $\xymatrix@-1pc{E\ar[r] & \spec A\ar@/_/[l]_e}$. Let $\hat
   \O_{E,e}=\varprojlim \O_E/\I_e^n\cong A[[T]]$. Choose $T$ such
that $b=(1+$higher
   terms$)\cdot dT$; this choice is unique up to $T\mapsto T+$higher
terms. We have that
   $f_*\I_e^n$ is locally free of rank $n$ for $n\ge 2$. Shrink a
little to make it free
   for $n=2$. Choose a basis $1,x$ for $f_*\I_e^2$ so that
$x=\frac{1}{T^2}(1+$higher$)$,
   and choose a basis $1,x,y$ for $f_*\I_e^3$ so that
$y=\frac{1}{T^3}(1+$higher$)$. Then
   you get that you can write $y^2+a_1xy+a_3y=x^3+a_2x^2+a_4x+a_6$.
Exercise: There is a
   unique choice of $y$ such that $a_1=a_3=0$ and a unique choice of
$x$ so that $a_2=0$.
   Thus, we get $y^2=x^3+a_4x+a_6$. Let $g_2=-4a_4$ and $g_3=-4a_6$.

   It remains to see what the action is. Let $u\in
\GG_m(A)=A^\times$. Then $T$ is
   replaced by $uT$, $x$ gets replaced by $u^{-2}x$, and $y$ gets
replaced by $u^{-3}y$.
   So $g_2$ gets replaced by $u^4g_2$ and $g_3$ gets replaced by
$u^6g_3$.
 \end{proof}
 Upshot: $\M_{1,1}\cong [Y/\GG_m]$.
