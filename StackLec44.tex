\sektion{44}{Brauer Groups and Gabber's Theorem}

Today let's go back to gerbes.

\underline{Brauer groups and quotient stacks}. Let $X$ be a scheme.
An \emph{Azumaya algebra} on $X$ is a locally free sheaf of
(non-commutative) $\O_X$-algebras $\A$ of finite rank such that
\'etale locally on $X$, $\A\cong End(\V)$ for some vector bundle $\V$
on $X$ (this implies that the center is $\O_X$). $\A\cong \A'$ if
there exist vector bundles $\V$ and $\V'$ such that $\A\otimes_{\O_X}
End(\V)\cong \A'\otimes_{\O_X}End{\V'}$. Giving a group structure of
tensor product over $\O_X$, we get the Brauer group $Br(X)$.

If you have an Azumaya algebra $\A$ of rank $n^2$ ($n$ is the rank of
that local vector bundle), then you can define
$P_\A=\isom(\A,M_{n\times n}(\O_X))$. This is a functor on
$X$-schemes, it gives you $(\sch/X)^{op}\to \set$, given by $(f:T\to
X)\mapsto $ the set of isomorphism $f^*\A\xrightarrow\sim M_{n\times
n}(\O_T)$. This $P_\A$ is a $PGL_n$-torsor (the Scholem-Noether
theorem tells you that all the automorphisms of $M_{n\times n}$ are
given by conjugation by some matrix). From a $PGL_n$-torsor, we get
the sequence
\[
  1\to \GG_m \to GL_n\to PGL_n\to 1
\]
from which we get a boundary map $H^1(X,PGL_n)\to H^2(X,\GG_m)$. This
sends a torsor to the stack $[P_\A/GL_n]$, which is $\GG_m$-gerbe. In
fact, this map lands in the torsion part of $H^2(X,\GG_m)$. The
reason is
\[\xymatrix{
  1\ar[r] & \GG_m \ar[r] & GL_n \ar[r] & PGL_n\ar[r] & 1\\
  1 \ar[r] & \mu_n \ar[r]\ar@{^(->}[u] & SL_n \ar[r]\ar@{^(->}[u]&
PGL_n \ar[r]\ar@{=}[u] & 1
}\]
So the map factors through $H^2(X,\mu_n)$ which is torsion.

We get a map $Br(X)\to H^2(X,\GG_m)_{tors}=:Br'(X)$ (cohomological
Brauer group) given by $\A\mapsto [P_\A/GL_n]$. We should check that
this is well-defined and a homomorphism, but we won't. It is not too
hard to check that this is injective. Under some circumstances, there
is no torsion \anton{normal and something?}
\begin{theorem}[Gabber]
  If $X$ has an ample sheaf (a little more general than
quasi-projective), then this map is an isomorphism.
\end{theorem}
Consider the piece of the long exact sequence
\[
  H^1(X,GL_n)\to H^1(X,PGL_n)\to H^2(X,\GG_m)
\]
The first map is $\V\mapsto P_{End(\V)}$. From that we see that the
kernel of the second map consists of \anton{}, which is the statement
of injectivity of the map $Br(X)\to Br'(X)$ (you also have to say
that if two Azumaya algebras are equivalent, then the torsors $P_\A$
are isomorphic: if $\A$ and $\A'$ with $P_\A\cong P_{\A'}$, then we
look at $P_\A^{-1}\wedge P_{\A'}$ and somehow show that
$\A\simeq\A'$; this shouldn't be hard). The second map is
$P_\A\mapsto \G_\A$.
\[\xymatrix{
  & \G_\A \ar[d]\\
  T \ar[r]_f\ar[ur]^{\tilde f} & X
}\]
$\G_\A(T)$ is the category of pairs $(\V_T,\iota)$ with $\V_T$ a
locally free rank $n$ on $T$ and $\iota:f^*\xrightarrow\sim
End(\V_T)$. This is a $\GG_m$-gerbe.

This means that there is a tautological locally free sheaf of rank
$n$ on this gerbe. $\G_\A$ comes equipped with a canonical locally
free sheaf $\V_\A$. To give a locally free sheaf on a stack is to
give a locally free sheaf for every morphism in and for every
diagram, an isomorphism of the two pull-backs. So take $\V_T$; if you
have a morphism $T\to T'$, then you automatically get an isomorphism
between $\V_T$ and the pullback of $\V_{T'}$.

Let $\G\to X$ be a $\GG_m$-gerbe, and let $\F$ be a quasi-coherent
sheaf on $\G$.
\begin{proposition}
  $\F$ has a canonical decomposition $\F\cong \bigoplus_{n\in \ZZ}
\F^{(n)}$.
\end{proposition}
If $f:T\to \G$, then you get a sheaf $\F_T:=f^*\F$. This sheaf has an
action of $\aut_f=\GG_m$, given by $u\in \GG_m$ acts by the
2-morphism given by $\mu$ over $\id_T:T\to T$ over $\G$. Then we get
a decomposition $\F_T\cong \bigoplus_{n\in \ZZ}\F_T^{(n)}$, with
$u\in \GG_m$ acts on $\F_{T}^{(n)}$ by multiplication by $u^n$.

\begin{definition}
  A \emph{twisted sheaf on $\G$} (the standard terminology is ``a
$\G$-twisted sheaf on $X$'') is a quasi-coherent sheaf $\F$ such that
$\F=\F^{(1)}$. You should really specify the character \dots since
we're working with $\GG_m$, we have the canonical character $(1)$.
\end{definition}
\begin{remark}
  $\V_\A$ on $\G_\A$ is twisted. Because we defined the action as
just multiplication.
\end{remark}
\begin{proposition}
  Let $\alpha\in H^2(X,\GG_m)$ with associated gerbe
$\pi:\G_\alpha\to X$. Then $\alpha\in Br(X)\subseteq H^2(X,\GG_m)$ if
and only if $\G_\alpha$ admits a twisted locally free sheaf.
\end{proposition}
Reason: $(\Rightarrow)$ is already done by remark. $(\Leftarrow)$ if
$\F_1=\F_1^{(n)}$ and $\F_2=\F_2^{(m)}$, then $\F_1\otimes \F_2 =
(\F\otimes \F)^{(n+m)}$ (if $u$ acts by $u^n$ and $u^m$, then it acts
on the tensor product by $u^{n+m}$). Also, $\F=\F^{(0)}$ if and only
if $\pi^*\pi_*\F\to \F$ is an isomorphism (so the ones that are
untwisted are just the sheaves on $X$).

Say $\V$ on $\G$ is twisted. Then $\V\otimes \check\V=(\V\otimes
\check \V)^{(0)}$ (because $u$ acts my multiplication by the inverse
on $\check\V$), so $\pi_*End(\V)=\A$ is a locally free sheaf on
$\O_X$-algebras on $X$ of finite rank. You check that if the gerbe is
trival, then this gives you endomorphisms of the vector bundle, so
this is an Azumaya algebra. You go through the definitions and check
that $\G_\A\cong \G_\alpha$.
\begin{remark}[Aside]
  Given an $\A$, you get $End(\V_\A)\cong \pi^* \A$ as part of the
data. If you start with $\pi:\G_\A\to X$, then you check that
$\pi_*End(\V_\A)\cong \A$.
\end{remark}
To understand this theorem of Gabber, the best thing to do is read de
Jong's proof.

Idea of Gabber's Theorem:\\
- Gabber's thesis is the case where $X$ is affine. If you search for
Gabber on MathSciNet, then there aren't many choices, so you should
find it easily.\\
- Given $X$ quasi-projective over $S$ and $\alpha\in
H^2(X,\GG_m)_{tors}$ you get $\G_\alpha\to X$. Zariski locally on $X$
there is a twisted vector bundle on $\G_\alpha$ (by the affine case).
We want to piece them together to get a global twisted sheaf. The
proof of this is on deJong's web page. We don't have time to do that.

Another way to think about this problem. What does it mean to have
such a vector bundle.
\begin{definition}
  A stack $\X$ over $S$ is a \emph{quotient stack} if $\X\cong [Z/G]$
where $Z$ is an algebraic space over $S$ and $G\subseteq GL_{n,S}$ is
a subgroup scheme flat over $S$.
\end{definition}
\begin{proposition}
  The following are equivalent for a stack $\X$.
  \begin{enumerate}
   \item $\X$ is a quotient stack.
   \item there is a vector bundle $\V$ on $\X$ so that for every
point $x:\spec k \to \X$, the action of the group scheme $\aut_x$ on
the fiber $\V_x$ is faithful.
  \end{enumerate}
\end{proposition}
Restatement: if $\alpha\in H^2(X,\GG_m)$, then $\alpha\in Br(X)$ if
and only if the gerbe $\G_\alpha$ is a quotient stack. A twisted
sheaf is an example of something where the automorphism group acts
faithfully: $\V_x = \bigoplus \V_{\bbar x}^{(a_i)}$ and the
$gcd(a_i)=1$. Certainly a twisted sheaf gives such a vector bundle,
but the other direction is also true.

The issue then is whether something is a quotient stack or not.
\begin{theorem}
  Let $k$ be a field, and let $\X$ over $k$ be a smooth
Deligne-Mumford stack with finite diagonal, and generic stabilizer is
trivial (there is an open set with no stabilizer). Then $\X$ is a
quotient stack.
\end{theorem}
This doesn't treat the problem because for a gerbe you often don't
have a generic stabilizer.

Idea of proof: you want to produce a vector bundle and there is not
much you can do. Let's look at the case where $k$ is characteristic
zero and algebraically closed. Let $T_\X$ be the dual of $\Om^1_\X$.
We claim first that it satisfies this proposition. You have the
coarse space $\X\to X$. Then we have
\[\xymatrix{
  [U/\Ga]\ar[d]\ar[r] & \X\ar[d]\\
  X'\ar[r] & X
}\]
where $\Ga$ is the stabilizer group of a point $x\in \X$. $\hat
U=\spec k[[t_1,\dots, t_r]]$ has an action of $\Ga$ on it.
$T_{\X,\hat U_x'}=\m/\m^2$. How do you get a decomposition when you
have a smooth thing: look at $\hat \O_{U,x}\to \hat \O_{U,x}/\m^2$
and find a section for $\m/\m^2$. Since we are in characteristic
zero, something is semi-simple representation of $\Ga$. So you get
$k[[\m/\m^2]]$ and something is trival because the action is fiathful.






