\sektion{38}{Keel-Mori}


\subsektion{Coarse moduli spaces}
Since the first day, we've talked about the $j$-invariant
$\M_{1,1}\to \AA^1_j$. (1) This is universal for maps to schemes in
the sense that for any $\M_{1,1}\to Z$, there is a unique
factorization through $\AA^1_j$. (2) if $k$ is an algebraically
closed field, then $|\M_{1,1}(k)|\xrightarrow{\sim} \AA^1_j(k)$.

In general, the analogue of $\AA^1_j$ will be called a \emph{coarse
moduli space}.

Say $S$ is a scheme and $\X$ is an algebraic stack over $S$
\anton{why not any stack? something about relative coarse moduli
spaces} with finite diagonal. Recall that $\Delta:\X\to \X\times_S
\X$ is finite if for any $(x_1,x_2):T\to \X\times_S \X$,
$\isom(x_1,x_2)\to T$ is finite (in particular, $\isom(x_1,x_2)$ is a
scheme). Quasi-finite is not enough, but maybe you can relax this
slightly to say that the $\aut_x$ are finite.

All the hypothesis in the paragraph above should go below the
definition.
\begin{definition}
  A \emph{coarse moduli space} for $\X$ over $S$ is a morphism
$\pi:\X\to X$ where $X$ is an algebraic space over $S$ such that
  \begin{enumerate}
   \item $\pi$ is universal for morphisms to algebraic spaces:
   \item if $k$ is algbraically closed, then the map $|\X(k)|\to
X(k)$ is bijective.\qedhere
  \end{enumerate}
\end{definition}
\begin{remark}
  $\pi:\X\to X$ is unique up to unique isomorphism (you only need
property 1 for this).
\end{remark}
\begin{theorem}[Keil-Mori (1997 in annals), Conrad (in his web page)]
  With the above assumptions, there exists a coarse moduli space
$\pi:\X\to X$ ($\X$ need not be Deligne-Mumford). Additionally,
  \begin{enumerate}
   \item $X$ is separated over $S$ and if $S$ is locally noetherian,
then $X$ is locally of finite type over $S$.
   \item $\pi$ is proper (we haven't defined this yet)
   \item if $X'\to X$ is a flat morphism of algebraic spaces, then
$\X'=\X\times_X X'\to X'$ is a coarse moduli space for $\X'$.
  \end{enumerate}
\end{theorem}
This theorem was folklore for a long time, then it was written
without proof in places. Keil and Mori proved it in the noetherian
case. Usually, once you prove something in the noetherian case, you
can get it in general, but there is no reason for that to be true
here because of the following warning.
\begin{warning}
  The formation of coarse moduli space \emph{does not} commute with
arbitrary base change, only with flat base change.
\end{warning}
What if you drop the finiteness assumption on the diagonal. You have
to throw out condition (2) \anton{for some reason}. For example, it
seems like the coarse moduli space for $[\AA^1_k/\GG_m]$ should be
$\spec k$. This satisfies condition (1), but the proof breaks
horribly.

We'll give the proof in the locally noetherian case. The proof in
general is along the same lines, but more technical.

Idea: start with your stack $\X$. First (we'll do it last) go to the
case where there exists a quasi-finite flat surjection $U\to \X$.
This isn't too bad (in \cite[III]{SGA}). The really subtle step is to
then go to the case where you have a \emph{finite} flat surjection
$U\to \X$. Back when we talked about Stein factorization, we had some
sort of argument like if we have $U\to X$ quasi-finite, you pick a
point $x\in X$ and you can find an \'etale neighborhood $V$ su that
$P\coprod U\to V$ is quasi-finite and finitely presented over $V$.
Then we reduce to the case where there exists a finite flat
surjection $\spec A_1\to \X$ (in \cite[III]{SGA}). This case, we've
essentially done. Many of the technical lemmas are also in
\cite[III]{SGA}.

What does it mean to be proper?
\begin{definition}
  If $f:\X\to Y$ is a morphism from an algebraic stack to a scheme,
then $f$ is \emph{closed} if for every closed substack $\Z\subseteq
\X$, the image of $\Z$ in $Y$ is closed. By the image, we mean take
all the field-valued points in $\Z$ and they give you field-valued
points in $Y$, and that set is the image; equivalently, cover $\Z$ by
a scheme and look at the image in $Y$. A morphism of algebraic stacks
$f:\X\to \Y$ is \emph{universally closed} if for every morphism $Y\to
\Y$ with $Y$ a scheme, the morphism $\X\times_\Y Y\to Y$ is closed.
$f:\X\to \Y$ is \emph{proper} if it is separated (diagonal is proper
\anton{does this actually make sense?}), finite type, and universally
closed.
\end{definition}
If $\pi:\X\to X$ is a coarse moduli space and $\pi$ is proper, then
$\pi^{-1}:OpenSet(X)\to OpenSubstack(\X)$ is bijective. Reason: say
$\U\subseteq \X$ is an open substack, then the claim is that
$\pi(\U)$ is open in $X$ and $\pi^{-1}(\pi(\U))=\U$. Complement of
$\U$: take a presentation $p:\tilde X\to \X$, then
\[\xymatrix{
  \tilde U'\ar@<.5ex>[d]\ar@<-.5ex>[d]\ar@{^(->}[r] & \tilde
X'\ar@<.5ex>[d]\ar@<-.5ex>[d]  & \tilde Z'\ar@{_(->}[l]
\ar@<.5ex>[d]\ar@<-.5ex>[d]\\
  \tilde U\ar@{^(->}[r]\ar[d] & \tilde X\ar[d]^p & \tilde
Z\ar@{_(->}[l] \ar@{.>}[d]\\
  \U\ar@{^(->}[r]\ar[d] & \X\ar[d] & \Z\ar@{_(.>}[l]\ar@{.>}[d]\\
  U\ar[r] & X & Z\ar@{_(->}[l]
}\]
where $Z$ is the complement of $U$ with the reduced structure.
Formation of maximal reduced subscheme commutes with smooth base
change, so $Z$ descends. $\pi$ proper so $\pi(\Z)\subseteq X$ is
closed. The claim is that $\pi^{-1}(\pi(\Z))_{red}=\Z$. This is
because $|\X(k)|\to X(k)$ is bijective. $\tilde U$ is an algebraic
space because $\U\to \X$ is representable.

In some sense, the coarse moduli space is a universal homeomorphism.
