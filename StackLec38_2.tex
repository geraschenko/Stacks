\subsektion{Proof: I}
$S$ is a scheme and $\X$ is an algebraic stack over $S$ which is
locally of finite presentation (forgot this last time, but it should
be there) with finite diagonal.
\begin{theorem}\label{lec38T:some}
  There exists a coarse moduli space over $S$ $\pi:\X\to X$ so that
all that stuff from last time \anton{}. Additionally,
  \begin{enumerate}
   \item $X$ is separated over $S$ and locally of finite type if $S$
is locally noetherian.
   \item $\pi$ is proper. (i.e.~that bijection of opens)
   \item if $X'\to X$ is flat, then $\X'=\X\times_X X'\to X'$ is a
coarse moduli space.
  \end{enumerate}
\end{theorem}
The typed up note for this are online. We only give the proof in the
locally noetherian case.

In the example $[\AA^1/\GG_m]$, something doesn't work. For $BG$, we
have that $BG\to \spec k$ is a coarse moduli space (because you only
do the testing on algebraically closed fields), but without these
special properties. Is there an example where the coarse moduli space
doesn't exist.

In general, if you have a quotient $[X/G]$, the ring of invariants
should be the coarse moduli space, but it might be infinitely
generated or something else bad.

\begin{remark}[Theorem is Zariski local on $\X$]
  If $\X=\bigcup \X_i$ where each $\X_i$ is open in $\X$ and each
$\X_i$ has a coarse moduli space $\X_i\to X_i$ as in theorem (with
the properties), then $\X$ also has a coarse moduli space as in
theorem. To see that, take the $\X_i$ and
  \[\xymatrix{
    \X_i\cap \X_j \ar@{^(->}[rr]\ar@{^(->}[dd] \ar[dr] & &
\X_j\ar[d]\\
    & X_{ij}\ar@{^(->}[r]\ar@{^(->}[d] & X_j\\
    \X_i\ar[r] & X_i
  }\qquad\qquad \xymatrix{
    \X_i\ar[r] \ar[d] & X_i\ar[d]\ar@/^4ex/[dd]\\
    \X\ar[r]\ar[dr] & X\ar@{-->}[d]_{\exists !}\\
    & Y
  }\]
  Then you glue to get $\X\to X$. Then we check that property (on the
right). Now check the other properties \anton{}.

  Note that this argument doesn't work in the \'etale topology; there
is no reason to expect $X_i\to X$ to be \'etale in that case.
\end{remark}
Thus, we can assume $S$ is a noetherian affine scheme.

\underline{Special case 1}: assume there exists a faithfully flat
surjection $\spec A_1\to \X$. Since the diagonal is finite, $\spec
A_1\times_\X \spec A_1=\spec A_2$ is affine (since finite over
affine). If you look at the proof from lecture 13 and change \'etale
to flat in some places, then something. You're really using that this
is really a groupoid. Set $A_0=Eq(A_1\rightrightarrows A_2)$, then
\begin{enumerate}
  \item[(a)] $A_0\to A_1$ is finite and integral. This implies that
$A_0$ is of finite type over $S$ \cite[7.8]{Atiyah-Macdonald}.
  \item[(b)] The topological space $|\spec A_0|$ is the topological
quotient of $|\spec A_1|$ by the equivalence relation defined by
$|\spec A_2|\to |\spec A_1\times \spec A_1|$.
  \item[(c)] $\pi:\X\to \spec A_0$ is universal for maps to schemes
\anton{$\X$ is presented by $\spec A_2\rightrightarrows \spec A_1$,
so the map $\pi$ is induced by the fact that $\spec A_1\to \spec A_0$
coequalizes those arrows}. We showed that $\spec A_0$ is the quotient
in the category of ringed spaces of $\spec A_1$ by $\spec A_2$. This
implies that $\pi_*\O_\X|_{(\spec A_0)_{et}}=\O_{\spec A_0}$. Just
look at maps to $\AA^1$. This is a general thing if you look at
coarse moduli space and push forward the sturcture sheaf you get the
structure sheaf.
\end{enumerate}
We need to check that $\pi$ is universal for maps to algebraic
spaces. So take an algebraic space $Y$. Then the sequence
\[
  Y(A_0)\to Y(A_1)\rightrightarrows Y(A_2)
\]
should be exact.

Fix an \'etale surjection $U\to Y$ with $U$ a scheme and set
$R=U\times_Y U$.

I guess we need to prove something first. The claim is that
$\hom(\X,Y)=Eq\bigl(Y(A_1)\rightrightarrows Y(A_2)\bigr)$
\[\xymatrix{
  \spec A_2 \ar@<.5ex>[d]\ar@<-.5ex>[d]\\
  \spec A_1\ar[d]\\
  \X \ar[r]\ar[dr] & F\ar[d]\\
  & Y
}\]
This would be clear if something \anton{} is \'etale cover. We need
to prove that $Y$ is a sheaf with respect to the fppf topology.
\anton{what?} $F$ is the presheaf of isomorphism classes in the
fibers of $\X$, so $\hom(\X,Y)=\hom(F,Y)$.

We should also correct the theorem. In the definition of coarse
moduli space, the universal property should be for morphisms to
quasi-separated algebraic spaces. The proof doesn't work for general
algebraic spaces; you'll see the reason.

\begin{theorem}
  Let $X$ be a quasi-separated algebraic space over $S$. Then $X$ is
a sheaf with respect to the fppf topology on $\sch/S$.
\end{theorem}
\begin{proof}
  Let $\bbar X$ be the sheaf (in the fppf topology) associated to $X$
and let $q:X\to \bbar X$ be the natural map of presheaves. First note
that $X$ is a separated presheaf, which is equivalent to injectivity
of $q$: If $s_1,s_2\in X(U)$, then $Z=U\times_{(s_1,s_2),X\times
X,\Delta} X$ is a scheme (because diagonal is representable).

  Surjectivity: say $s\in \bbar X(U)$. By injectivity, to show that
$s$ is in the image of $X$, we may replace $U$ by an open cover.
Thus, we can assume $U$ is quasi-compact. Now we can write $X=\bigcup
X_i$ where each $X_i$ is quasi-compact. Now let $\bbar X_i$ be the
fppf sheaf associated to $X_i$. It is an exercise to check that
$X(U)=\varprojlim X_i(U)$ and $\bbar X(U)=\varprojlim \bbar X_i(U)$
because $U$ is quasi-compact.

  So we may assume $X$ is quasi-compact. Let $X_0\to X$ be an \'etale
sujection with $X_0$ quasi-compact scheme. The claim is that $\bbar
X_0\times_{\bbar X,s} U$ ($\bbar X_0=X_0$ and $\bbar U=U$) is a
scheme which is \'etale and quasi-compact over $U$. This would do it
because lifting to $X_0$ is even better than lifting to $X$.

  The quasi-compact is there to be able to do descent theory for
quasi-affine maps: an \'etale and quasi-compact map is quasi-affine.
We may then work fppf locally on $U$. Thus, we can assume there
exists an $\tilde s\in X(U)$ with $q(\tilde s)=s$. Then by
injectivity, $\bbar X_0\times_{\bbar X,s} \bbar U=X_0\times_{X,\tilde
s} U$, which is \'etale over $U$ (because $X_0$ is \'etale over $U$)
and it is quasi-compact because
  \[\xymatrix{
    X_0\times_X U\ar[d] \ar[r] & X_0\times_S U\ar[r]\ar[d] & U\\
    X\ar[r]^\Delta & X\times X
  }\]
\end{proof}
Now let's go back to the sequence
\[
  Y(A_0)\xrightarrow{j} Y(A_1)\rightrightarrows Y(A_2).
\]
Let's do injectivity of the first map. As before, $U\to Y$ is an
\'etale surjection from a scheme. Let $\eta_1,\eta_2\in Y(A_0)$ such
taht $j(\eta_1)=j(\eta_2)$. First point: we can replace $\spec A_0$
by an \'etale cover $A_0\to R_0$.
\[\def\oA#1{\smash{\underset{A_{#1}}{\,\otimes\,}}}
 \xymatrix{
  Y(A_0)\ar[d]\ar[r] & Y(A_1)\ar[d]\ar@<.5ex>[r]\ar@<-.5ex>[r] &
Y(A_2)\ar[d]\\
  Y(R_0)\ar@<.5ex>[d]\ar@<-.5ex>[d]\ar[r] & Y(R_0\otimes_{A_0} A_1)
\ar@<.5ex>[d]\ar@<-.5ex>[d] \ar@<.5ex>[r]\ar@<-.5ex>[r] &
Y(T_0\otimes_{A_0} A_2)\ar@<.5ex>[d]\ar@<-.5ex>[d]\\
  Y(R_0\oA0 R_0) \ar[r] & Y(R_0\oA0 A_1)\oA1 Y(R_0\oA0 A_1)
\ar@<.5ex>[r]\ar@<-.5ex>[r] & Y(R_0\oA0 A_2)\oA2 Y(R_0\oA0 A_2)
}\]
\anton{what do those tensor products over $A_1$ and $A_2$ mean?}
Let $R=U\times_Y U\subseteq U\times U$. We want to show that $(\tilde
\eta_1,\tilde \eta_2)\in R(A_0)$. Because of the universal property
for schemes, we have exactness of the right column.
\[\xymatrix{
  (\tilde \eta_1, \tilde \eta_2)\ar@{}[r]|{\mbox{$\in$}} \ar@{|->}[d]
&R(A_0)\ar[d]\\
  U(A_1)\times U(A_1)\ar@<.5ex>[d]\ar@<-.5ex>[d] &
R(A_1)\ar@<.5ex>[d]\ar@<-.5ex>[d] \ar@{_(->}[l]\\
  U(A_2)\times U(A_2) & R(A_2)\ar@{_(->}[l]
}\]

Now exactness in the middle:
\[\xymatrix{
  \spec A_2\ar@<.5ex>[r]\ar@<-.5ex>[r] & \spec A_1\ar[r]
\ar@/^2ex/[rr]^\eta & \spec A_0 \ar@{-->}[r] & Y
}\]
$\eta$ the map $\spec A_1\to Y$. We can work fppf locally on $\spec
A_0$. We can even assume $A_0$ is strictly henselian local ring
(maybe works??). Choose a point $\bbar x:\spec k\to \spec A_0$ where
$k$ is separably closed. Then let $I_{\bbar x}$ be the category of
all \'etale maps $U\to \spec A_0$ which $\bbar x$ factors through.
Then $\O_{\spec A_0,\bbar x} = \lim_{I_{\bbar x}} \Ga(U,\O_U)$.
Properties:
\begin{enumerate}
  \item $\O_{\spec A_0,\bbar x}$ is local
  \item If $R\to k$ where $R$ is strictly henselian and local and
$\spec \O_{\spec A_0,\bbar x}\to k$, then there is a unique morphism
$\O_{\spec A_0,\bbar x}\to R$.
\end{enumerate}
\[\xymatrix{
  \spec A_2\ar@<.5ex>[r]\ar@<-.5ex>[r] & \spec A_1\ar[r] & \spec
A_0\ar[r] & Y\\
  & \prod \spec R_i\ar[u]\ar[r] & \spec \O_{\spec A_0,\bbar x}\ar[u]
}\]
Then $\spec R_i\to \X$ faithfully flat surjective and $\X\to \spec
\O_{\spec A_0,\bbar x}$. Something about if coarse moduli space then
doesn't matter which flat surjection. Reduce to the case when $A_0$
strictly henselian local rings.
