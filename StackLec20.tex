

 \bigskip
 %%%% %%%% %%%% %%%% %%%% %%%% %%%% %%%% %%%% %%%% %%%%
 \subsektion{Underlying topological space of an algebraic space}

 There is a topological space associated to an algebraic space. Let $X$ be an algebraic
 space over a scheme $S$. A point of $X$ is a monomorphism $i:\spec k\to X$ with $k$ a
 field. You shouldn't confuse this with a geometric point (where we take $k$ to be a
 separably closed field), this is just a monomorphism of sheaves. If we take $k$ to be
 separably closed field, it usually won't be a monomorphism of sheaves! You could drop
 the monomorphism assumption if you change the equivalence relation to say there is a
 third thing they map to. \anton{this definition would make some things easier}
 \begin{example}
   Let $X$ be a scheme. If we have any morphism $\spec k\to X$, this gives a point $x\in
   X$ and embedding $k(x)\hookrightarrow k$. That is, the morphism always factors through
   $\spec k(x)$. Given an inclusion of fields $k\hookrightarrow k'$, when is $\spec k'\to
   \spec k$ a monomorphism? If and only if $k\to k'$ is an isomorphism \anton{exercise in
   Galois theory}  What about inseparable extensions? To be a monomorphism, it must be a
   monomorphism after base change. Look at $k'\to k'\otimes_k k'$.

   Consider $k(t)$ and $k(t^{1/p})$, then we have $k(x)\to
   k(t)[x,y]/(x^p=y^p=1)=k(x)[u]/(u^p=0)$ with $u=x-y$. We get two maps
   $k(x)[u]/(u^p=0)\to k(x)[u]/(u^p=0)$, given by $u\mapsto u$ and $u\mapsto 0$, which
   induce the same map. Therefore, it isn't a monomorphism.
 \end{example}
 We say that $i_1:\spec k_1\to X$ is equivalent to $i_2:\spec k_2\to X$ if there exists a
 diagram
 \[\xymatrix{
    \spec k_1 \ar[r]^\sigma_{\sim} \ar[d]_{i_1} & \spec k_2 \ar[dl]^{i_2}\\
    X
 }\]

 Then we take $|X|$ to be the set of points modulo equivalence. If $Y\subseteq X$ is a
 closed subspace, then you get an inclusion $|Y|\subseteq |X|$, and we define the
 topology on $|X|$ by declaring these subsets to be closed.

 Key point: $|X|$ is functorial in $X$. This is not clear. If we have $X\to Y$ and $\spec
 k\to X$ a monomorphism, there is no reason $\spec k\to Y$ should be a monomorphism. The
 following lemma gives it to us. One then checks that the map is continuous.
 \begin{lemma}
   Let $f:\spec k\to Y$ be a map of algebraic spaces with $k$ a field. Then there exists
   a unique (up the equivalence relation) point $i:\spec k'\to Y$ and factorization
   \[
    f: \spec k\xrightarrow g \spec k'\xrightarrow i Y
   \]
 \end{lemma}
 \begin{proof}
   This is annoyingly subtle. How do you know there are any points in $Y$ at all. You
   know there is a dense open which has points, but blah.

   First note that we can assume $Y$ is quasi-compact: Take any \'etale cover $U\to Y$,
   then take a point in $U_k$ and see where it goes in $U$, and take a quasi-compact open
   subset of $U$ and quotient by the relation. Choose $U\to Y$ an \'etale surjection with
   $U$ a quasi-compact scheme. Let $Z_p$ be the disjoint union of spectra of residue
   fields of images of $\spec k\times_Y U\to U$. Since $U$ is quasi-compact and \'etale,
   the first is a finite union of points. Similarly, let $R_p$ be the disjoint union of
   spectra of residue fields of images of $\spec k\times_Y (U\times_Y U)\rightrightarrows
   U\times_Y U$. Then we see that $R_p=Z_1\times_{U,p_1} (U\times_Y U)$. So $R_p$ is a
   finite \'etale equivalence relation on $Z_p$ (in fact, it is the induced relation).
   Thus, the quotient $Z_p/R_p$, which is a scheme. We get a monomorphism
   $Z_p/R_p\hookrightarrow Y$, and my construction we get a flat surjection $\spec k\to
   Z_p/R_p$, so $Z_p/R_p$ is a spectrum of a field.
 \end{proof}

 $f:X\to Y$ is proper if it is of finite type, separated, and universally closed
 (i.e.~for every $Y'\to Y$, the map $|Y\times_Y Y'|\to |Y'|$ is a closed map). Check from
 the definition that this agrees with the working definition:
 \begin{lemma}
   If $X\to Y$ is separated of finite presentation and $X'\to X$ is a proper
   representable surjection, then $X\to Y$ is proper if and only if $X'\to Y$ is proper.
 \end{lemma}

 If $X$ is given to you as a functor, the only way to check properness is with the
 valuative criterion.
