
\subsektion{Proof: IV}

Recall that we have an Artin stack $\X/S$ of finite type with finite
diagonal (and we were doing the case where $S$ is noetherian). We
sketched how to construct a coarse moduli space.

\begin{theorem}[C]
  $\X=\bigcup \X_i$ with each $\X_i$ admitting a quasi-finite flat
covering.
\end{theorem}
This is the only theorem we haven't talked about \dots it is a slice
argument in \cite[3]{SGA} and we've done something similar earlier.

Then take a really nice \'etale cover $\W$ of $\X$ which has a coarse
space, and show that $\W\times_\X \W$ has a coarse space which is an
\'etale equivalence relation.

Today we'll talk more about the construction of the coarse space.

\begin{theorem}
  Let $\X/S$ finite type with finite diagonal and assume $\X$ is
Deligne-Mumford. Then for every geometric point $\bbar x:\spec k\to
\X$ ($k$ separably closed), there exists an \'etale neighborhood
$X'\to X$ of the image of $\bbar x$ in the coarse space $X$ such that
$\X\times_X X'\cong [U/\Ga_{\bbar x}]$ where $U$ is a finite
$X'$-scheme and $\Ga_{\bbar x}=\aut_{\X(k)}(\bbar x)$. This is why
people say DM stacks are like orbifolds \dots they locally look like
the quotient by the stabilizer group at that point.
\end{theorem}
\begin{proof}
  We can assume $\X$ is quasi-compact (everything is local on the
coares space, and if the coarse space is quasi-compact, then so is
$\X$). Choose an \'etale quasi-compact surjection $U\to \X$. Then
$U\to X$ is quasi-finite, separated, and of finite type, so we can
choose a base change $X'\to X$ an \'etale neighborhood of $\bbar x$
(this includes a lifting of $\bbar x$ to $X'$) such that $U\times_X
X'=P\sqcup T$ such that $P\to X'$ is finite and $T\times_{X'}\spec
k=\varnothing$.

  Replace $X$ by $X'$. Replacing $\X$ by the image of $P$, we can
assume there is a finite \'etale covering $P\to \X\to X$. $\X\to X$
is proper surjective and $P\to X$ is proper, thne $P\to \X$ is proper
(finite for some reason \anton{$P\to X'$ is finite}). The lemma for
schemes: $Z_1\xrightarrow f Z_2 \xrightarrow Z_3$ with $g$ separated
and surjective and $gf$ proper, then $f$ is proper.

  \[\xymatrix{
    \coprod P^i_{\bbar x} \ar[r]\ar[d]& \X_{sh} \ar[d]\ar[r] & \spec
\O_{X,\bbar x}^{sh}\ar[d]\\
    P\ar[r] & \X\ar[r] & X 
  }\]
  with $P^i_{\bbar x}$ strictly hensilian and local. So after further
replacing $X$ by an \'etale covering, we can assume that we only get
one $P_{\bbar x}:=P\times_X \spec \O_{X,\bbar x}^{sh}$ which is
strictly hensilian and local.
  \[\xymatrix{
    \coprod_\Ga P_{\bbar x}=R_{\bbar x}\ar@<.5ex>[r]\ar@<-.5ex>[r]
&P_{\bbar x} \ar[r] & \X_{sh} \ar[r] & \spec \O_{X,\bbar x}^{sh}
  }\]
  Looking at just closed points, a closed point in $R_{\bbar x}$ is
an automorphism of a point in $P_{\bbar x}$. You see the group
structure from the groupoid structure.

  The group structure on $\Ga_{\bbar x}$. We have
  \[
   \bigl(\coprod_{\Ga_{\bbar x}} P_{\bbar
x}\bigr)\times_{p_2,P_{\bbar x}, p_1}\bigl(\coprod_{\Ga_{\bbar x}}
P_{\bbar x}\bigr) \to \bigl(\coprod_{\Ga_{\bbar x}} P_{\bbar x}\bigr)
  \]
  The second projection $R_{\bbar x}\xrightarrow{p_2} P_{\bbar x}$
defines an action of $\Ga_{\bbar x}$ on $P_{\bbar x}$ which doesn't
have to act trivially.

  So we see that $[P_{\bbar x}/\Ga_{\bbar x}]=\X_{sh}\to \spec
\O_{X,\bbar x}^{sh}$. Something is obtained by some limit, so somehow
you get the theorem.
\end{proof}
\begin{remark}
  If $\X\to X$ is \'etale, then \'etale locally on $X$ (with notation
as above), $\X=X\times B\Ga$ for some finite group $\Ga$. If the map
$\X_{sh}\to \spec \O_{X,\bbar x}^{sh}$ is \'etale, then $P_{\bbar
x}\to \spec \O_{X,\bbar x}^{sh}$ is an isomorphism (something about
strictly hensilian local rings) so the two projections must be equal.
\end{remark}
\begin{example}
  $\M_{1,1,S}$, then the coarse space is $\AA^1_{j,S}$ (even if 6 is
not invertible). Note that we haven't actually shown yet that
$\M_{1,1,S}$ has finite diagonal, so we don't yet know that the
coarse space exists. Let's first show finite diagonal. Take elliptic
curves with some level structure, then it will be finite flat over
$\M_{1,1,S}$. In the case when $3\in \O_S^\times$ (then take the open
subset where 2 is invertible and glue), $V=\Spec_S
(\O_S[\mu,\w][1/(\mu^3-1)]/(\w^2+\w+1))$. This classifies elliptic
curves $E$ with an isomorphism $\sigma: (\ZZ/3)^2\xrightarrow\sim
E[3]$ (the 3-torsion of $E$). The universal curve over it is
$X^3+Y^3+Z^3=3\mu XYZ$, and the basis for the 3-torsion is
$p=[1,0,-1]$, $q=[-1,\w,0]$.

  We have $V\to \M_{1,1,S}=[V/GL_2(\FF_3)]$. Now we clearly have
finite diagonal. This implies that the coarse space $M_{1,1,S}$ is
relative spectrum over $S$ of the $GL_2(\FF_3)$-invariants in $\O_V$.
I think its pretty hard to show that this is $\AA^1_{j,S}$ (by the
way, $j=\bigl(3\mu(\mu^3-2^3)/(\mu^3-1)\bigr)^3$). Let's do it
another way.

  Consider the case when $S=\spec k$. Then $\pi:\M_{1,1,k}\to
M_{1,1,k}$ and we have a map $j:\M_{1,1,k}\to \AA^1_k$, so we get a
map $h:M_{1,1,k}\to \AA^1_k$. We want $h$ to be an isomorphism. We
know that $M_{1,1,k}$ is a geometrically normal curve because you are
taking invariants of a smooth curve by a finite group \'etale
locally, so the quotient is normal. $h$ induces a bijection on
$\Om$-valued points for every algebraically closed field $\Om$. That
implies that $h$ is an isomorphism.

  Again. (something about coarse spaces commuting with flat base
change. You can see that $M_{1,1,k}$ is proper over $\AA^1$ because
$V$ is finite over the line. We know that $j$ and $\pi$ are proper,
$\pi$ is surjective, which implies that $h$ is proper, and for some
reason finite. Since coarse spaces commute with flat base change, we
can assume $k=\bar k$.

  Something is bad
  \[\xymatrix{
    k(j^{1/p})[\e]/\e^p \ar[r]\ar[d] & k(j^{1/p})\ar[d]\\
    k(j^{1/p})\ar[r] & k(j)
  }\]
  \anton{somebody should explain this to me}

  Now take the case where we have an artinian local ring $R$ with
residue field $k$ and let $S=\spec R$. Say $J\subseteq R$ is a square
zero ideal which is annihilated by the maximal ideal of $R$, and say
$R_0=R/J$. Then we have over $\M_{1,1,R}$
  \[
    J\otimes \O_{\M_{1,1,k}}\to \O_{\M_{1,1,R}}\to
\O_{\M_{1,1,R_0}}\to 0
  \]
  But we get exactness on the left by tensoring the following with
$\O_{\M_{1,1,R}}$
  \[
   0\to J\to R\to R_0\to 0
  \]
  Now on $\AA^1_R$ we get a sequence
  \[
   0\to J\otimes_k j_*\O_{\M_{1,1,k}}\to j_*\O_{\M_{1,1,R}} \to
j_*\O_{\M_{1,1,R_0}}
  \]
  \[\xymatrix{
    0\ar[r] & J\otimes_k j_*\O_{\M_{1,1,k}}\ar[r] &
j_*\O_{\M_{1,1,R}} \ar[r] & j_*\O_{\M_{1,1,R_0}} \\
    0\ar[r] & J\otimes \O_{\AA^1_k} \ar[u]\ar[r] & \O_{\AA^1_R}
\ar[u]\ar[r] & \O_{\AA^1_{R_0}} \ar[r] & 0
  }\]
  By induction we can assume the outer two are isos so the middle one
is an iso.

  In general to check that $\O_{\AA^1_{j,S}}\to j_*\O_{\M_{1,1,S}}$
is an isomorphism reduce to the case of artinian local.



\end{example}
$\M_{g}$ also has finite diagonal. If $R$ is a complete discrete
valuation ring and $C_1,C_2\in \M_g(R)$, then any isomorphism
$\sigma_\eta:C_{1,\eta}\xrightarrow\sim C_{2,\eta}$ ($\eta$ the
generic point) over $\spec k$ extends uniquely to an isomorphism
$C_1\xrightarrow\sim C_2$. One way to do this is to say $C_1$ is the
minimal regular model of the generic guy.

