\subsektion{More Cohomological Descent}

Recall the setup. We have $\T_\udot\xrightarrow\e \T_\varnothing$,
and we want to compute $R^i\e_*F_\udot$.

If $I\in Ab(\T_\udot)$, then (1) each $I_k\in Ab(\T_k)$ is injective,
and (2) $\e_*I$ (concentrated in degree zero) is quasi-isomorphic to
$\e_*^0I_0 \to \e_*^1 I_1\to \cdots$, so there is no higher
cohomology. By Lang, we have $E^{pq}_1=R^q\e^p_*F_p\Rightarrow
R^{p+q}\e_*F$.

\begin{example}[\v Cech cohomology]
 What happens in the \v Cech cohomology situation. Let $X$ be a
quasi-compact separated scheme, and let $X=\bigcup U_i$ a finite
covering with each $U_i$ affine. Let $\F$ be quasi-coherent on $X$.
Then the simplicial scheme we get is $\coprod U_i\to X$. The next
step is $\coprod U_i\cap U_j$, then triple intersections (allowing
$i=j$, by the way; this is how you get the sections); call this thing
$U_\udot$. We're supposed to have a spectral sequence
$E^{pq}_1=H^q(\coprod_{i_0\dots i_p}U_{i_0\dots i_p}, \F)\Rightarrow
H^{p+q}(U_\udot  \F)$. The picture is

 \anton{pictures}

 For $q>0$, the whole thing is zero because you have an affine
scheme. The bottom row is the \v Cech complex, except for this $i<j$
issue. You get the complex by taking alternating sums. This
simplicial abelian group (of 0th cohomologies) can be handled in 2
ways; one of them (the normalized complex) is the usual \v Cech
complex, and the other is what we have. There is some result that
says that they are the same thing (give the same cohomlogy).
\end{example}

Let's say why injective objects look like what we've claimed they
are. If you have an abelian sheaf in $Ab(\T_\udot)$. Then for every
$n$, we have a restriction map $r_n:Ab(\T_\udot)\to Ab(\T_n)$. This
is an exact functor (this is how you define exact sequences in
$Ab(\T_\udot)$. It has a right adjoint $e_n:Ab(\T_n)\to
Ab(\T_\udot)$, given by $(e_n F)_k= \prod_{\rho\in
\hom_\Delta([k],[n])} \rho_* F$ ($\rho$ gives a map $\T_n\to \T_k$).
Any functor with an exact left adjoint takes injectives to
injectives, so $e_n$ does so. Thus, $e_n$ of an injective is
injective.

Given $F_\udot \in Ab(\T_\udot)$, choose for every $n$ and inclusion
$F_n\hookrightarrow I_n$ where $I_n\in Ab(\T_n)$ is injective. Then
$F_\udot \hookrightarrow \prod_n e_n(f_n(F_\udot))\hookrightarrow
\prod_n e_n(I_n)$. A corollary of this is that every injective sheaf
in $Ab(\T_\udot)$ is a direct summand of a sheaf of the form $\prod_n
e_n(I_n)$, with the $I_n$ injective for each $n$.

$\rho_*$ takes injectives to injectives (because exact left adjoint).
Thus, each $e_n(I_n)$ is injective at each level, and since each
injective is a summand of one of these, we've checked the first
point. Also, it is possible to show this by showing that $r_n$ has an
exact left adjoint, so it takes injectives to injectives.

To check the second point for a direct sum, it is equivalent to check
it for each summand. To check that $\e_*^0I_0\to \e_*^1I_1\to\cdots$
has no higher cohomology, it is enough to consider $I_\udot = \prod_n
e_n(I_n)$. Let $T_n:\Delta\to Ab$ be given by $[p]\mapsto
\prod_{\hom_\Delta([p],[n])}\ZZ$. Let $\tilde T_n$ be the associated
total complex. In this case, the sequence with the $\e_*^i$s is
$(\e_*^nI_n)\otimes_\ZZ \tilde T$; you first push down to $[n]$ and
then push down to $\varnothing$. But $\tilde T_n$ is a complex of
abelian groups, it computes cellular homology of the standard
$n$-simplex, which is zero, so the tensor product is isomorphic to
$\e_*^n I_n$.

Let's now take $\X$ to be an algebraic stack. Let $X\to \X$ be a
smooth surjection with $X$ and algebraic space. Then we get
$X_\udot$, which is a simplicial algebraic space. Now you have to be
a little careful. Remember we're thinking about $\X_\liset$. We'd
like to say we get an augmentation $X_{\udot \liset}\to \X_\liset$,
but the lisse-\'etale topos is not functorial; for example
$\Delta:X\to X\times_\X X$ doesn't give a morphism of topoi. However,
we do get $X_{\udot \liset}^+\to \X_\liset$ (the $+$ means restrict
to $\Delta^+$); somehow all these maps are smooth, so we're ok. We
also have morphisms of topoi $X_{\udot \liset}^+\to X_{\udot et}^+$
and $X_{\udot et}\to X_{\udot et}^+$. A sheaf of
$\O_{\X_\udot}$-modules in $X_{\udot et}$ (for each $i$, the sheaf on
$X_i$ is an $\O_{X_i}$-modules in a compatible way) is quasi-coherent
if each $F_i$ on $X_i$ is quasi-coherent and the transition maps
$X(\sigma)^\star F_i\to F_j$ is an isomorphism for every $\sigma$ a
morphism in $\Delta$. Similarly, we can talk about quasi-coherent
sheaves on the other topoi.
\begin{theorem}
  $\qco(X_{\udot \liset}^+)$, $\qco(\X)$, $\qco(X_{\udot et}^+)$, and
$\qco(X_{\udot et})$ are all equivalent (by the maps above), and the
map $\qco(\X)\to \qco(X_{\udot et})$ is the natural restriction.
\end{theorem}
Note that at least for this subject, you don't need to worry about
the degeneracy maps.

The good statement is that there is an equivalence of derived
categories $\D^+_{qcoh}(\X)\to \D^+_{qco}(X_{\udot et})$. Concretely,
you can always choose presentations
\[\xymatrix{
  X_\udot \ar[r]^{f_\udot}\ar[d] & Y_\udot\ar[d]\\
  \X\ar[r]^f & \Y
}\qquad\qquad\xymatrix{
 \D(X_{\udot et})\ar[d]^\wr \ar[r]^{Rf_{\udot*}} & D(Y_{\udot
et})\ar[d]^\wr\\
\D^+_{qcoh}(\X)\ar[r]^{Rf_*} & \D^+_{qcoh}(\Y)
}\]
The upshot is (cohomological descent) that you can compute the
cohomology of a stack by a spectral sequence
$E^{pq}_1=H^q(X_p,F|_{X_p})\Rightarrow H^{p+q}(\X,F)$. This is the
general tool for computing cohomology of artin stacks.
\begin{example}
  Let $\X=B_kG$ (let's say $G$ is a finite group). Then we have
  \[\xymatrix{
    G\times G \ar@3{->}[r] & G\ar@2{->}[r] & \spec k \ar[r] & B_kG\\
    Map(G\times G,F) & Map(G,F)\ar@3{->}[l] & F\ar@2{->}[l]
  }\]
  (and the other arrows). If $F$ is a sheaf on $BG$ (i.e.~a
representation of $G$), then we get the second row. That bottom row
is the standard complex computing group cohomology. The derived
functors of invariants are computed by this second row.
\end{example}
So cohomology of a stack is a mixture of group cohomology and regular
old cohomology.
\begin{example}[special case]
  Take $G=\ZZ/p$ and $k=\FF_p$. Then $H^i(G,k)\cong \FF_p$ for each
$i\ge 0$. So we see that $H^*(BG,\O_{BG})$ is unbounded.
\end{example}
\begin{definition}
  Let $\X$ be a Deligne-Mumford stack over some noetherian $S$ with
finite diagonal. We call $\X$ \emph{tame} if for every algebraically
closed field $k$ and $x\in \X(k)$, the order of $\aut_{\X(k)}(x)$ is
invertible in $k$.
\end{definition}
Think of $\M_{1,1}$ outside of characteristic 2 and 3.

In this case, look at the coarse space $\pi:\X\to X$.
\begin{proposition}
  For any quasi-coherent sheaf $\F$ on $\X$, $R^i\pi_*\F=0$ for $i>0$
\end{proposition}
\begin{corollary}
  $H^p(\X,\F)=H^p(X,\pi_*\F)$.
\end{corollary}
In particular, for an algebraic space, there is an integer so that
cohomology vanishes after a certain point \anton{we proved it by
taking a dense open scheme and do some kind of induction\dots there
should be some assumption like finite type over a field}

\begin{proof}[Proof of Proposition]
  The result is \'etale local on $X$. We can assume $\X=[U/\Ga]$
where $U\to \X$ is finite and $\Ga$ is the stabilizer group of some
point. The assumption that it is tame means that $\Ga$ has order
which is invertible in $\X$ (again taking an \'etale map if needed).
Then a quasi-coherent sheaf on $\X$ is the same thing as an
$\O_U$-module $F$ with a lifting of the $\Ga$ action to $F$.

  We have $(M,$action$)\mapsto M^\Ga$, which is an $A$-module. So
we're really just computing group cohomology. But group cohomology
over a field where the order of the group is invertible is zero for
higher stuff. $\frac{1}{|\Ga|}\sum_{\gamma\in\Ga} \gamma$ is a
projector from $M$ to $M^\Ga$.
\end{proof}

Often when you do algebraic geometry, you'll see stamements like
``quotient singularities are as good as smooth''. Quotient
singularity means you're locally the quotient of a smooth thing by a
finite group. If $X$ has quotient singularities, it means that it
looks like the coarse space of a smooth DM stack. It is clear that
the coarse space of a smooth DM stack has only quotient
singularities. If you have only quotient singularities and you're
over $\QQ$, then there is some DM stack whose coarse space is the
thing you started with. You can prove lots of things about $\X$ from
its coarse space. In characteristic $p$, this is unknown; bummer.

The problem is: given an $X/k$ with $k$ even algebraically closed of
characteristic $p$, with quotient singularities, produce a smooth DM
stack with $X$ as its coarse space. This would be very interesting if
somebody solves it.
