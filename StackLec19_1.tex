
 
 \subsektion{Higher Direct Images of Proper Maps}

 \begin{lemma}
   Let $f:X\to Y$ be a morphism of topoi and assume that $\ab_X$ and $\ab_Y$ have
   enough injectives (this always holds, actually). Then there is a spectral sequence
   $E^{p,q}_2=H^p(Y,R^q\!f_*F)\Rightarrow H^{p+q}(X,F)$ for all $F\in \ab_X$.
 \end{lemma}
 \begin{proof}
   $R^q\!f_*$ is the $q$-th derived functor of $f_*:\ab_X\to \ab_Y$, and $H^p$ is the
   $p$-th derived functor of the global sections functor. Observe that we have a
   factorization as shown, and $f_*$ takes injective objects to injective objects (Lemma
   \ref{lec18L:preserve_injectives}).
   \[\xymatrix{
    \ab_X \ar[r]^{f_*} \ar[dr]_{\Ga(X,-)} & \ab_Y \ar[d]^{\Ga(Y,-)}\\
    & \ab
   }\]
    There is a very general theorem (the \emph{Grothendieck spectral sequence},
   \cite[XX.9.6]{Lang:Algebra}) that says that whenever you have a factorization $G\circ
   F$ of a functor so that $F$ takes injective objects to $G$-acyclic objects, then you
   get a spectral sequence $R^pG(R^qF)\Rightarrow R^{p+q}(G\circ F)$.
 \end{proof}
 \begin{lemma}
   Let $f:X\to Y$ be an affine morphism of algebraic spaces. Then for any quasi-coherent
   sheaf $F$ on $X$ and $q>0$ we have that $R^q\!f_*F=0$, where we think of
   $f_*:\ab_{X_{et}}\to \ab_{Y_{et}}$.
 \end{lemma}
 \begin{proof}
   Given $q>0$, consider the following statments.
   \begin{enumerate}
     \item[$(*)_q$] For every affine morphism $f:X\to Y$ of algebraic spaces and $F$
     quasi-coherent on $X$, $R^s\!f_*F=0$ for all $1\le s\le q$.
     \item[$(*)_q'$] For every affine scheme $X$ and every $F$ quasi-coherent on $X$,
     $H^s(X_{et},F)=0$ for $1\le s\le q$.
   \end{enumerate}
   We claim that $(*)_q'$ implies $(*)_q$. This is because $R^s\!f_*F$ is the sheaf
   associated to the presheaf $(Y'\to Y)\mapsto H^s(X\times_Y Y',F|_{X\times_Y Y'})$
   \anton{exercise}. Thus it is enough to consider the case when $Y$ is affine, so
   $(*)_q'\Rightarrow (*)_q$.

   Now we prove that $(*)_q'$ holds by induction on $q$. For $q=1$, we are looking at
   $H^1(X_{et},F)=Ext^1(\O_X,F)$ with $F$ quasi-coherent and $X$ an affine scheme. You
   can think of the elements as isomorphism classes of extensions of $\O_X$-modules
   \[
    0\to F\to E\to \O_X\to 0.
   \]
   The point is that $\hom(\O_X,-)\cong \Ga(X,-)$, so they give the same derived functors
   (whether you compute in $\O_X\mod$ or in $\ab$). Now the result is clear because $F$
   and $\O_X$ are quasi-coherent, so $E$ is quasi-coherent. By the equivalence of
   categories $\qco(X_{zar})\cong \qco(X_{et})$, we know that all such sequences are
   split because we can compute the Zariski cohomology to be zero.

   So we've proven $(*)_1'$, and therefore $(*)_1$. Now assume $(*)_q$. This implies that
   for every affine morphism $f:X\to Y$ and every $F$ quasi-coherent on $X$, we have an
   inclusion $H^{q+1}(Y,f_*F)\hookrightarrow H^{q+1}(X,F)$. To see this, look at the
   spectral sequence $E_2^{s,t}=H^s(Y,R^t\!f_*F)\Rightarrow H^{s+t}(X,F)$.
   \[\begin{xy}
     (0,5) *+!UR{t}, (10,0) *+!UR{s},
     (8,0) *+!D{H^{q+1}(Y,f_*F)}, (0,4) *++!L{H^0(Y,R^{q+1}f_*F)},
     (0,3.5); (10,3.5) **@{.}; (6.5,3.5); (6.5,0) **@{.},
     (3.25,2.1) *{\mbox{\Huge 0}},
     (0,.7);(10,.7) **@{.},
     (1.5,.35) *{\mbox{\Large $\ast$}},
     (3.8,.35) *{\mbox{\Large $\cdots$}},
     (6.15,.35) *{\mbox{\Large $\ast$}},
     (0,.35) *++!R{0}, (0,2.1) *++!R{\mbox{\Large $\vdots$}},
     (0,3.2) *++!R{q}, (0,4) *+!R{q+1},
     (1.5,0) *++!U{0}, (3.8,0) *++!U{\mbox{\Large $\cdots$}},
     (6.15,0) *++!U{q}, (8,0) *+!U{q+1},
     \ar 0;(0,5)
     \ar 0;(10,0)
   \end{xy}\]
   There is a natural map $H^{q+1}(Y,f_*F)\to H^{q+1}(x,F)$, and the kernel is the stuff
   killed of as you run the spectral sequence. But you can see that you aren't killing
   anything off because of all the zeros.

   Now we'd like to show that $(*)_q\Rightarrow (*)_{q+1}'$. Take a class $\alpha\in
   H^{q+1}(X_{et},F)$. We'd like to say that it is zero. To compute, we take an injective
   resolution, take global sections and then take cohomology. There exists an \'etale
   surjection $g:U\to X$ with $U$ an affine scheme such that $\alpha\mapsto 0\in
   H^{q+1}(U_{et},F)$ \anton{This is because we have a resolution of sheaves!}. $F\mapsto
   I^\udot$. $\tilde \alpha\in \Ga(X,I^{q+1})$ locally because this is a resolution.

   Now consider the sequence
   \[
    0\to F \to g_*g^* F\to Q\to 0.
   \]
   Then we get a long exact sequence
   \[
    \underbrace{H^q(X,Q)}_{=0\text{ by }(*)_q'}\to H^{q+1}(X,F)\to H^{q+1}(X,g_*g^*F)
    \hookrightarrow H^{q+1}(U,g^*F)
   \]
   and $\alpha\mapsto 0$ in $H^{q+1}(U,g^*F)$, so it is zero.
 \end{proof}
 \begin{lemma}
   Let $X$ be a scheme, and let $\e:X_{et}\to X_{zar}$ be the natural morphism of topoi
   (coming from the inclusion of sites). Then for every quasi-coherent sheaf $F$ on
   $X_{et}$ we have $R^q\e_*F=0$ for all $q>0$.
 \end{lemma}
 \begin{proof}
   $R^q\e_*F$ is the \'etale sheaf associated to the presheaf on $X_{zar}$ given by
   $U\mapsto H^q(U_{et},F)$. It is enough to check on affine open subsets, and there they
   are zero.
 \end{proof}
 \begin{lemma}
   Let $X$ be a quasi-compact separated algebraic space. Then $\qco(X)$ has enough
   injectives and for every injective $I\in \qco(X)$, we have $H^q(X,I)=0$ for $q>0$.
 \end{lemma}
 \begin{proof}
   Let $\pi:U\to X$ be a quasi-compact \'etale surjection with $U$ a disjoint union of
   affine schemes. By the case of schemes, $\qco(U)$ (we can be ambiguous about the
   topology because they are equivalent) has enough injectives, and for injective $I_U\in
   \qco(U)$, $\pi_*I_U$ is injective in $\qco(X)$ (since $\pi_*$ has an exact left
   adjoint).

   So for any $F\in \qco(X)$, choose an injection $\pi^*F\hookrightarrow I_U$, with
   $I_U\in \qco(U)$ injective. Then you get
   \[
    F\hookrightarrow \pi_*\pi^* \hookrightarrow \pi_*I_U
   \]
   In fact, by taking $F$ to be injective, we see that any injective in $\qco(X)$ is a
   direct summand of $\pi_*I_U$ for some injective $I_U\in \qco(U)$

   So to prove the second statement, it is enough to note that for an injective $I\in
   \qco(U)$, $H^q(X,\pi_*I)=H^q(U_{et},I)=H^q(U_{zar},I)=0$. The first equality is
   because we have a spectral sequence $E_2^{p,q}=H^p(X,R^q\pi_* I)\Rightarrow
   H^{p+q}(U,I)$. All of the $R^q\pi_*I=0$ for $q>0$ are zero because $X$ is separated,
   so $\pi:U\to X$ is an affine morphism.
 \end{proof}
 \begin{lemma}
   Let $f:X\to Y$ be a quasi-compact and separated morphism of algebraic spaces. Then for
   every quasi-coherent $F$ on $X$, the sheaves $R^q\!f_*F$ are quasi-coherent on $Y$.
 \end{lemma}
 \begin{proof}
   We're looking at the sheaf associated to the presheaf $(Y'\to Y)\mapsto H^q(X\times_Y
   Y',F)$. To check quasi-coherence, we can look \'etale locally on $Y$, so we can assume
   $Y$ is an affine scheme. So $\qco(X)$ has enough injectives. If $F$ happens to be
   injective in $\qco(X)$, the $H^q(X\times_Y Y',F)$ are zero. Thus, to compute the
   cohomology, we can choose an injective resolution over $X$ and push it forward. But we
   know that push-forwards of quasi-coherent sheaves are quasi-coherent. Choose an
   injective resolution $F\to I^\udot$ in $\qco(X)$. Then $R^q\!f_*F=\mathcal
   H^q(f_*I^0\to f_*I^1\to \cdots)$.
 \end{proof}
 \begin{theorem}
   If $f:X\to Y$ is a proper morphism of locally noetherian algebraic spaces and $F$ is
   coherent on $X$, then $R^q\!f_*F$ are coherent on $Y$.
 \end{theorem}
 Recall that for now proper just means that \anton{}
 \begin{proof}
   Coherence is local on $Y$, so we can assume $Y$ is an affine scheme. Thus, there
   exists a proper birational morphism $X'\to X$ so that $X'\to Y$ is projective.

   Next we need that for every integral closed subspace of $X$, there is a coherent sheaf
   supported on it.
 \end{proof}
