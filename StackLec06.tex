\sektion{6}{Representable functors are fppf sheaves}

% I'll be out of town Wednesday and Friday, so there won't be any lectures. We'll try to
% make them up later. The discussion should still happen (to talk about problem set 2, I
% suppose).

The main result for today is Theorem  \ref{lec06T:hXfppfsheaf}, that representable functors are sheaves in the fppf topology.

Recall the following three propositions. The first one is standard, and the other two are in EGA.
\begin{proposition}
 Let $A\to B$ be a ring morphism. Then the following are equivalent.
 \begin{enumerate}
  \item $A\to B$ is faithfully flat.
  \item A sequence of $A$-modules $M'\to M\to M''$ is exact if and only if $M'\otimes_A B\to M\otimes_A B\to M''\otimes_A B$ is exact.
  \item A homomorphism $M'\to M$ of $A$-modules is injective if and only if $M'\otimes_A B\to M\otimes_A B$ is injective.
  \item $B$ is flat over $A$ and $\bigl(M\otimes_A B=0 \Longrightarrow M=0\bigr)$.
 \end{enumerate}
\end{proposition}
\begin{proposition}[{\cite[IV.1.10.4]{EGA}}]\label{lec06P:flfp=>open}
 A flat morphism that is locally of finite presentation is open.
\end{proposition}
\begin{proposition}[{\cite[\anton{somewhere in IV}]{EGA}}]\label{lec06P:ffqc_induced_topology}
 If $f:X\to Y$ is faithfully flat and quasi-compact, then a subset $U\subseteq Y$ is open if and only if $f^{-1}(U)\subseteq X$ is open (i.e.~$Y$ has the induced topology).
\end{proposition}
The following corollary allows us to deal fppf morphisms.
\begin{corollary}\label{lec06C:fppf->qcmpt_cover}
 Let $f:X\to Y$ be faithfully flat and locally of finite presentation, and let $Y=\bigcup_i U_i$ be a Zariski open covering, with each $U_i$ affine. Then for each $i$, there is a Zariski covering $f^{-1}(U_i)=\bigcup_j V_{ij}$ with $V_{ij}$ quasi-compact and $f(V_{ij})=U_i$.
\end{corollary}
\begin{proof}
 Let $p\in f^{-1}(U_i)$, and let $W_{ip}\subseteq f^{-1}(U_i)$ be an affine open neighborhood of $p$. Given any open affine set $W_{iq}\subseteq f^{-1}(U_i)$, $f(W_{iq})$ is open by proposition \ref{lec06P:flfp=>open}. Since $U_i$ is affine, it is quasi-compact, so we can choose a finite set $\{W_{iq_k}\}_{k=1}^n$ so that the $f(W_{ip})\cup \bigcup_k f(W_{iq_k})=U_i$. Now we can define $V_{ip}:= W_{ip}\cup \bigcup_k W_{iq_k}$. Since $V_{ip}$ is a finite union of affines, it is quasi-compact. Furthermore, $p\in V_{ip}$, so $\{V_{ip}\}_{p\in f^{-1}(U_i)}$ covers $f^{-1}(U_i)$.
\end{proof}
\anton{a flat morphism satisfying the conclusion of this corollary is said to be \emph{fpqc}}
% \begin{remark}[Dfn of fppf topology]
%   If $S$ is a scheme, then take objects $S$-schemes, with morphisms $S$-morphisms.
%   $\{X_i\to X\}\in Cov(X)$ if each $X_i\to X$ is flat and locally of finite
%   presentation, and $\coprod X_i\to X$ is surjective. You can check that if you only
%   work with affine schemes over $S$, then you get the same topos, so that might be a
%   good way to avoid technical issues.
% \end{remark}
\begin{proposition}\label{lec06P:fflat_exact_sequence}
 If $f:A\to B$ is faithfully flat, then the following sequence is exact.
 \[\xymatrix{
  A\ar[r]^f & B \ar@<-.5ex>[rr]_<>(.5){b\mapsto 1\otimes b} \ar@<.5ex>[rr]^<>(.5){b\mapsto
 b\otimes 1} & & B\otimes_A B }\]
\end{proposition}
\begin{proof}
 We use the following trick. Since $B$ is faithfully flat over $A$, exactness of the sequence in question is equivalent to exactness of the sequence obtained by tensoring with $B$.
 \[\xymatrix{
 B\ar[rr]^<>(.5){b'\mapsto 1\otimes b'} & &
 B\otimes_A B \ar@<.5ex>[rrr]^<>(.5){b\otimes b'\mapsto b\otimes 1\otimes b'}
              \ar@<-.5ex>[rrr]_<>(.5){b\otimes b'\mapsto 1\otimes b\otimes b'} & & &
 B\otimes_A B\otimes_A B
 }\]
 The first map is injective because the multiplication map $m:B\otimes_A B\to B$ is a section. Now we check exactness in the middle: if the two maps from $B\otimes_A B$ agree on some element, $\sum b_i\otimes b'_i$, then we have $\sum b_i\otimes 1\otimes b'_i= \sum 1\otimes b_i\otimes b'_i$. Applying $1\otimes m$, we get $\sum b_i\otimes b'_i = \sum 1\otimes b_ib'_i = 1\otimes \sum_i b_ib'_i$, so the original element $\sum b_i\otimes b'_i$ is in the image of the first map, proving exactness in the middle.
%   To check exactness in the middle, we use the following trick. For any faithfully flat
%   morphism $A\to A'$, we have the following diagram.
%   \[\xymatrix@R-1pc{
%    A\otimes_A A' \ar@{}[d]|{\wr\parallel} \ar[r]
%    & B\otimes_A A'\ar@<.5ex>[r] \ar@<-.5ex>[r] \ar@{}[d]|{\wr\parallel}& (B\otimes_A B)\otimes_A A' \ar@{}[d]|{\wr\parallel}\\
%    A'\ar[r] & B\otimes_A A' \ar@<.5ex>[r] \ar@<-.5ex>[r] & (B\otimes_A A')\otimes_{A'}(B\otimes_A
%    A') }
%   \]
%    Since $A'$ is faithfully flat over $A$, exactness of the sequence in question is
%   equivalent to exactness of the top row, which is equivalent to exactness of the bottom
%   row.
%
%   Thus, you can assume there is a $g:B\to A$ such that $A\to B\xrightarrow{g} A$ is
%   the identity (take $A'=B$, then we get $B\to B\otimes_A B\xrightarrow{\Delta}
%   B$).\anton{what?}
%
%   exactness in the middle: Let $b\in B$ with $e_1(b)=1\otimes b = b\otimes 1=e_2(b)$. We
%   claim that $b=fg(b)$. Inside $B\otimes_A B$, we have assumed that $1\otimes b=b\otimes
%   1$. Applying $g\otimes 1$ to get to $A\otimes_A B\cong B$ (via $f$), these elements
%   are sent to $b$ and $fg(b)$, respectively, so $fg(b)=b$.
\end{proof}
\begin{remark}[Faithfully flat extensions] \label{lec06R:fflat_extensions}
 We will use the above trick again. The fact that we could tensor with $B$ and get multiplication, as section of $B\to B\otimes_A B$, effectively makes it so that we can \emph{assume} we have a section of $A\to B$. Sometimes, we'll base extend by an fppf cover and get a section for some map, but to save writing we'll say that you can assume the original map has a section. Here is how the above argument would look:

 Base extending by $B$, we may assume we have a section $g$ of $f$; in particular, $f$ is injective. Let $b\in B$ with $1\otimes b = b\otimes 1$. Applying $fg\otimes \id$, we have that $b=fg(b)$, so $b$ is in the image of $f$, proving exactness in the middle.
\end{remark}
\begin{corollary}\label{lec06C:fflat_affine_condition}
 If $V\to U$ is a faithfully flat map of affine schemes, and $X$ is an affine scheme, then the sequence $h_X(U)\to h_X(V)\rightrightarrows h_X(V\times_U V)$ is exact.
\end{corollary}
\begin{proof}
 Let $U=\spec A$, $V=\spec B$, and $X=\spec R$. By Proposition \ref{lec06P:fflat_exact_sequence}, we have the exact sequence $A\to B\rightrightarrows B\otimes_A B$, and we wish to show exactness of the sequence $\hom(R,A)\to \hom(R,B) \rightrightarrows \hom(R,B\otimes_A B)$.

 Since $A$ injects into $B$, two maps from $R$ to $A$ which agree in $B$ are the same. If $f:R\to B$ satisfies $1\otimes f(r)=f(r)\otimes 1$, then $f(r)$ lies in $A$, so $f$ is obtained from a map $R\to A$.
\end{proof}
\begin{lemma}\label{lec06L:zariski+affine_condition=>fppf_sheaf}
 Let $F:\sch^{op}\to \set$ be a presheaf satisfying the following conditions.
 \begin{enumerate}
  \item $F$ is a sheaf in the big Zariski topology.
  \item If $V\to U$ is a faithfully flat morphism of affine schemes, then the sequence $F(U)\to F(V)\rightrightarrows F(V\times_U V)$ is exact.
 \end{enumerate}
 Then $F$ is a sheaf for the fppf topology.
\end{lemma}
\anton{if you remove the word ``faithfully'' from the second condition, then you can conclude that $F$ is a sheaf in the fpqc topology}
\begin{remark}
  The following proof also works if you work over some scheme $X$.
\end{remark}
\begin{proof}[Proof of Lemma \ref{lec06L:zariski+affine_condition=>fppf_sheaf}]
  Let $\{U_i\to U\}\in Cov_{fppf}(U)$, and let $V=\coprod U_i$, then we have the diagram
  \[\xymatrix@R-1pc{
    F(U)\ar[r] \ar@{}[d]|{\parallel} & F(V) \ar@<.5ex>[r] \ar@<-.5ex>[r] \ar[d]^(.45){\wr} &
    F(V\times_U V) \ar[d]^(.45)\wr\\
    F(U) \ar[r] & \prod_i F(U_i) \ar@<.5ex>[r] \ar@<-.5ex>[r]& \prod_{i,j} F(U_i\times_U
    U_j)
  }\]
   where the vertical isomorphisms follow from the fact that $F$ is a Zariski sheaf.
  Thus, it is enough to consider coverings consisting of a single morphism $\{V\to U\}$.

  \underline{Note}: If $\{U_i\to U\}\in Cov(U)$ is a finite set of maps with $U_i$ and
  $U$ affine, then $V=\coprod U_i$ is also affine. The top sequence is exact by
  assumption, so the sheaf condition (exactness of the bottom sequence) is verified.

  For a general (single element) fppf covering $f:V\to U$, choose a Zariski cover
  $V=\bigcup V_i$ with $V_i$ quasi-compact and $f(V_i)=U_i$ affine (we can do this by
  Corollary \ref{lec06C:fppf->qcmpt_cover}). Write each $V_i=\bigcup_a V_{ia}$ as a
  finite union of affines. Then consider the following diagram.
  \[\hspace*{-3ex}
  \xymatrix@!0 @R=3.5pc @C=10pc{
   F(U)\ar[r]^<>(.5)\gamma \ar[d]^\delta
   & F(V) \ar@<.5ex>[r] \ar@<-.5ex>[r] \ar[d]_\varepsilon
   & F(V\times_UV) \ar[d]\\
   \prod_iF(U_i) \ar@<.5ex>[d] \ar@<-.5ex>[d] \ar[r]^<>(.5)\beta
   & \prod_i \prod_a F(V_{ia})\ar@<.5ex>[d] \ar@<-.5ex>[d] \ar@<.5ex>[r] \ar@<-.5ex>[r]
   & \prod_{i} \prod_{a,b} F(V_{ia}\times_U V_{ib})\\
   \prod_{i,j} F(U_i\cap U_j) \ar[r]^<>(.5)\alpha
   & \prod_{i,j}\prod_{a,b} F(V_{ia}\cap V_{jb})
  }\quad
  \xymatrix@!0 @R=3.5pc @C=4pc{
   h\ar@{|->}[d] \ar@{|.>}[r]
   & *++[o][F-]{x} \ar@{|->}[d] \ar@/^/@{|->}[r] \ar@/_/@{|->}[r]
   & y \ar@{|->}[d] \\
   e \ar@{|->}p+(-.3,-1) \ar@{|->}p+(.3,-1) \ar@{|->}[r]
   & d \ar@/^/@{|.>}[r] \ar@/^/@{|.>}[r] \ar@/_/@{|.>}[r] \ar@/_/@{|->}[d] \ar@/^/@{|->}[d]
   & c\\
   f=f' \ar@{|.>}[r] & g
  }\hspace*{-2em}\]
  The $U_i$ cover $U$ and the $V_{ia}$ cover $V$ in the usual Zariski sense. Since $F$
  is a Zariski sheaf, the two vertical columns are exact. For a fixed $i$, the $V_{ia}$
  are a finite number of affines which cover the affine $U_i$, so by the note, the
  middle horizontal sequence is exact. We wish to show that the top sequence is exact.

  Since $\beta\circ \delta$ is injective, we must have that $\gamma$ is injective.
  Observe that this shows that $F$ is separated in the fppf topology. In particular,
  since $\{V_{ia}\cap V_{jb}\to U_i\cap U_j\}_{a,b}\in Cov_{fppf}(U_i\cap U_j)$,
  $\alpha$ must be injective!

  Now we check exactness at $F(V)$ by a diagram chase, illustrated above.\footnote{\label{lec06:footnote_on_chasing}The
  starting object is circled. A solid arrow indicates that an object at one end is
  defined by the object at the other end. A dotted arrow indicates that commutativity of
  the diagram forces the two objects to be related as described. For example, we know
  (by commutativity of the diagram) that both horizontal maps take $d$ to $c$;
  therefore, by exactness of the middle row, $d$ defines the element $e$.} Let $x\in
  F(V)$ be taken to $y\in F(V\times_U V)$ by both maps. Then $d$ must be taken by both
  maps to $c$, so by exactness of the middle row, it comes from some $e$. Since $d$ is
  the image of $x$, it is taken to some $g$ by both maps. The two images $f$ and $f'$
  must both be taken to $g$; since $\alpha$ is injective, we must have $f=f'$.
  Therefore, $e$ must be the image of some $h\in F(U)$. Since $\varepsilon$ is injective
  and $\varepsilon(x)=d=\varepsilon\circ \gamma(h)$, we get $\gamma(h)=x$.
\end{proof}
\begin{theorem}\label{lec06T:hXfppfsheaf}
  Let $X$ be a scheme. Then $h_X:\sch^{op}\to \set$ is a sheaf for the fppf
  topology.
\end{theorem}
\begin{proof}
  (Affine case) First assume that $X$ is affine. Combining Lemma
  \ref{lec06L:zariski+affine_condition=>fppf_sheaf} with Corollary
  \ref{lec06C:fflat_affine_condition}, we have that $h_X$ is a sheaf in the fppf
  topology.

  (General case) Now let $X$ be any scheme. Write $X=\bigcup X_i$ as a union of open
  affine subschemes. By Lemma \ref{lec06L:zariski+affine_condition=>fppf_sheaf}, it is
  enough to consider an fppf covering of the form $t:V\to U$, where $U$ and $V$ are
  affine.

  We have to check the exactness of
  \[\xymatrix{
  h_X(U)\ar[r]^\alpha & h_X(V)\ar@<.5ex>[r] \ar@<-.5ex>[r] & h_X(V\times_U V).
  }\]
   First we do injectivity. Suppose $f,g\in h_X(U)$ are identified by $\alpha$. Then we
  have $\xymatrix@-1pc{V\ar[r]^t & U \ar@<.5ex>[r]^f \ar@<-.5ex>[r]_g & X}$ with
  $ft=gt$. In particular, the maps of sets must agree; since $t$ is surjective, $f$ and
  $g$ must be set-theoretically equal. Now consider $U_i=f^{-1}(X_i)=g^{-1}(X_i)$. By
  the affine case, $f|_{U_i}=g|_{U_i}$ scheme-theoretically. Therefore, we get $f=g$.

  Now we check exactness in the middle. We will denote the forgetful functor to $\sch\to
  \textbf{Top}$ by $|\cdot|$. Let $f\in h_X(V)$ with $fp_1=fp_2$: $\xymatrix@-.5pc{
  V\times_U V \ar@<.5ex>[r]^<>(.5){p_1} \ar@<-.5ex>[r]_<>(.5){p_2} & V\ar[r]^f& X}$.
  Applying the forgetful functor, we get the diagram
  \[\xymatrix{
   |V\times_U V| \ar@{-->}[r] \ar@(ur,u)[rr]^{|p_1|} \ar@(dr,d)[rr]_{|p_2|}
   & |V|\times_{|U|}|V| \ar@<.5ex>[r]^<>(.5){\pi_1} \ar@<-.5ex>[r]_<>(.5){\pi_2}
   & |V| \ar[dr]_{|t|} \ar[r]^{|f|} & |X|\\
   & & & |U|\ar@{.>}[u]_{h}
  }\]
   The dashed arrow exists by the universal property of $|V|\times_{|U|}|V|$. For some
  reason, we have that $|f| \pi_1=|f| \pi_2$ \anton{why?}. If $v_1$ and $v_2$ are
  two points in $V$ which lie over the same point in $U$, then $(v_1,v_2)\in
  |V|\times_{|U|}|V|$, and we get $f(v_1)=|f| \pi_1(v_1,v_2)=|f|
  \pi_2(v_1,v_2)=f(v_2)$. Thus, we get a well-defined map $h:|U|\to |X|$ given by
  $u\mapsto f\bigl(t^{-1}(u)\bigr)$. By Proposition \ref{lec06P:ffqc_induced_topology},
  the topology on $U$ is induced by $t$, so $h$ is continuous.

  Let $V_i=f^{-1}(X_i)$ and $U_i=h^{-1}(X_i)$. Then $V_i\to U_i$ are fppf coverings. By
  the affine case, we have (unique) morphisms of schemes $h_i:U_i\to X_i$ so that
  $f|_{V_i}=h_i\circ t|_{V_i}$. Covering the intersections $X_i\cap X_j$ by affines and
  using the uniqueness, we have that the $h_i$ agree on intersections $U_i\cap U_j$.
  Therefore, we get a morphism of schemes $h:U\to X$ so that $f=h\circ t$.
%
%   i.e.~defines the same map $V\times_U V\to X$. $U$ has the induced topology, so you get
%   a continuous map $h:|U|\to X$. Now we have to define a map on structure sheaves.
%   $U_i=h^{-1}(X_i)\to X_i$ is an honest morphism by the affine case.
%   \[
%      f^{-1}(X_i)\times_{U_i} f^{-1}(X_i)\rightrightarrows f^{-1}(X_i) \to U_i \to X
%   \]
%   Over each $U_i$, we get a map $h^{-1}\O_X|_{U_i}\to \O_{U_i}$. By uniqueness, they agree
%   on overlaps, so we get a map of ringed spaces $(U,\O_U)\to (X,\O_X)$. Checking locally,
%   we see that it is a map of schemes.
 \end{proof}
