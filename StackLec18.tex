\sektion{18}{Chow's Lemma.}


 Today we'll prove Chow's lemma for algebraic spaces. In the interest of time, we'll
 cheat by quoting the following big theorem from scheme theory, and showing that it works
 for algebraic spaces.
 \begin{theorem}[{\cite[5.2.2]{Raynaud-Gruson}}] \label{lec18T:RGblow-up}
   Let $X$ be a quasi-compact and quasi-separated scheme, $U\subseteq P$ a quasi-compact
   open subscheme of a scheme $P$, and $f:X\to P$ a morphism of schemes of finite
   presentation which is flat over $U$. Then there exists a blow-up $P'\to P$ supported
   on $P\setminus U$ such that the strict transform of $X$, $X'\to P'$, is flat.
 \end{theorem}
 The \emph{strict transform} $X'$ is the scheme-theoretic closure of the pre-image of $U$
 (i.e.~of the morphism $X\times_P U\hookrightarrow X\times_P P'$). Note that strict
 transform makes sense for algebraic spaces as well.
 \begin{proof}[Proof when $X$ is an algebraic space.]
   Let $\tilde X\to X$ be an \'etale cover, with $\tilde X$ a quasi-compact separated
   $P$-scheme. Let $P'\to P$ be a blow-up supported on $P\setminus U$, and let $\tilde
   X'$ be the strict transform of $\tilde X$.
   \[\xymatrix{
      \tilde X\times_P U \ar[r]\ar[d] & \tilde X\times_P P'\ar[d]^{et}\\
      X\times_P U \ar[r] & X\times_P P'
   }\qquad\qquad\qquad
   \xymatrix{
      \tilde X'\ar[d] \ar[r]\ar@{}[dr]|(.25)\pb & X' \ar[d]\ar[r] & P'\ar[d]\\
      \tilde X\ar[r] & X\ar[r] & P
   }\]
    The diagram on the left is cartesian (simply by abstract non-sense). Scheme-theoretic
   closure commutes with \'etale base change \anton{basically because quasi-coherent
   sheaves of ideals pull back, right?}, so the strict transform $\tilde X$ is isomorphic
   to the product of the strict transform $X'$ with $\tilde X$ over $X$. That is, the
   left square of the right diagram is cartesian. Now applying the theorem in the case of
   schemes, we get that $\tilde X'\to P'$ is flat. But flatness is \'etale local and
   $\tilde X'\to X'$ is an \'etale cover, so $X'\to P'$ is flat, as desired.
 \end{proof}

 \begin{theorem}[Chow's Lemma]
   Let $S$ be a quasi-compact scheme and $X$ a separated algebraic space of finite
   presentation over $S$ (in particular, $X$ is quasi-compact). Then there exists a proper
   birational $S$-morphism from a scheme $X'\to X$ with $X'$ quasi-projective over $S$.
 \end{theorem}
 Note that proper is a stable property morphisms (by what \cite{Vistoli} claims is
 \cite[IV.2.7.1]{EGA}), so it makes sense for a morphism from a scheme to an algebraic
 space to be proper. The map is birational in the sense of Remark
 \ref{lec12R:non-stable_properties} (birationality descends along \'etale covers).

 \begin{proof}
   By Corollary \ref{lec14C:sep_algsp->dense_subscheme}, we may choose a dense
   open subspace $U\subseteq X$ with $U$ a scheme. By Chow's lemma for schemes, we may
   assume $U$ is quasi-projective over $S$ (maybe by shrinking) \anton{why is $U\to S$
   quasi-compact?}. Thus, we have immersions $U\hookrightarrow X$ and
   $U\hookrightarrow \PP^n_S$, so we get an immersion $U\hookrightarrow \PP^n_S\times_S
   X$.
  % \[\xymatrix{
  %    U\ar@{^(->}[r]\ar@{^(->}[d] & X\\
  %    \PP^n_S
  % }\qquad\qquad
  % \xymatrix{
  %    & X\times_S \PP^n_S\\
  %    U\ar@{^(->}[r] \ar@{^(->}[ur] & X
  % }\]
   \[\xymatrix{
      & & U \ar@{^(->}[d] \ar@{^(->}[dr]\\
      X'\ar[r] & X_1 \ar[r]^{\text{strict}}_{\text{trans}} \ar[d]^{\pi'}_{\text{flat}} &
      X_0\ar[d]_{\pi}\ar@{^(->}[r]^<>(.5){cl} &
      \PP^n_S\times_S X \ar@{}[dr]|(.25)\pb \ar[d]\ar[r] & X\ar[d]\\
      & P'\ar[r]^{\text{blow}}_{\text{up}} & P\ar@{^(->}[r]^{cl} & \PP^n_S \ar[r] & S
   }\]
    Let $X_0$ be the scheme-theoretic closure of $U$ in $X\times_S \PP^n_S$, and let $P$
   be the scheme-theoretic closure of $U$ in $\PP^n_S$. Since $\PP^n_S\times_S X\to X$ is
   proper and closed immersions are proper, $X_0\to X$ is a proper birational map
   (birational because $X_0$ and $X$ are both birational to $U$). Moreover, note that
   $P\to S$ is projective. Both $X_0$ and $P$ are birationally equivalent to $U$, so
   $\pi$ is birational. Since $\pi$ is an isomorphism on the image of $U$, it is flat
   there, so we may apply Theorem \ref{lec18T:RGblow-up} to get a blow-up $P'\to P$
   supported on $P\setminus U$ so that the strict transform $X_1$ of $X_0$ is flat over
   $P'$. Note that $X_1\to X_0\times_P P'$ is a closed immersion, so it is proper, and
   $X_0\times_P P'\to X_0$ is proper (because $P'\to P$ is proper), so $X_1\to X_0$ is
   proper, so $X_1\to X$ is proper and birational.

   On the pre-image of $U$, $\pi'$ is the same as $\pi$ (an isomorphism). Since
   $\pi'$ is flat, the fibers of $X_1$ over $P'$ are all zero-dimensional, so they are all
   finite (since $X_1\to P'$ is of finite presentation\anton{}). By Corollary
   \ref{lec16C:sep,lqfin,lfp_over_scheme=scheme}, $X_1$ is a scheme. By Chow's lemma for
   schemes, \anton{have we verified the hypotheses?} There is some scheme $X'$
   quasi-projective over $S$, with $X'\to X_1$ proper birational. Then note that $X'\to
   X$ is also proper birational.
 \end{proof}
 \begin{remark}
   You can get away without applying Chow's lemma for schemes a second time. Blow-ups are
   projective (since they are defined as Proj of a sheaf of graded algebras), so $P'\to
   S$ is projective. We have that $\pi':X_1\to P'$ is flat and birational. One can check
   that this implies that $\pi'$ is an open immersion, so it is quasi-projective. Thus,
   we could have actually taken $X'=X_1$.
 \end{remark}

% So after blowing up on $P$, we can assume that $\pi:X\to P$ is also flat. This implies
% that that $\pi$ is an open immersion: it is an exercise if you know that $X$ is a
% scheme; to check that $X$ is a scheme, it is enough to check that $X\to P$ is
% quasi-finite \anton{by that corollary!}. So let $\tilde X\to X$ be an \'etale cover with
% $\tilde X$ quasi-compact and separated over $P$. Then $\tilde X\to P$ is again a flat
% morphism of finite presentation of schemes which is quasi-finite over a dense open in
% $P$ (such a morphism must be quasi-finite because you know how the dimensions of the
% fibers behave under flat morphisms).  This implies that $\tilde X\to P$ is quasi-finite,
% which implies that $\pi$ is quasi-finite. \anton{say back in lec 16 that finite means
% locally of finite presentation.} Thus, $X$ is a scheme.
%
% By the way, once you know it is a scheme, you can just apply Chow's lemma for schemes.

 What you can prove for schemes which are not quasi-projective always involves using
 Chow's lemma to reduce to the quasi-projective case. This kind of shows that everything
 you can do for schemes, you can do for algebraic spaces.
