\sektion{35}{The Lisse-\'etale site on an algebraic stack}


\underline{Quasi-coherent sheaves}. We have to define a good topology
on a stack where the sheaves will live. Let $\X$ be an algebraic
stack over $S$, then we want a good topos of sheaves on $\X$.
\begin{itemize}
 \item \'etale. This is OK for Deligne-Mumford stacks (and is clearly
the good topology). In general, if $\X$ is an algebraic stack, you
might try to take objects to be $\Y\to \X$ representable \'etale
morphisms of algebraic stacks. Then any morphism between two such
things something. Morphisms are $\X$-morphisms. This gives a site,
but it doesn't have enough sheaves.

 Let $k=\bbar k$ and let $G/k$ be a connected smooth group scheme
(like $G=\GG_m$). Then $BG$ has no non-trivial (non-identity) \'etale
covers. Reason: assume $\Y\to BG$ is \'etale, then
 \[\xymatrix{
  Y\ar[r]\ar[d]_{et} & \Y\ar[d]^{et}\\
  \spec k \ar[r] & BG
 }\]
 Then $Y=\coprod \spec k$ with an action of $G$. Since $G$ is
connected, this is the trivial action. This implies that $\Y$ is a
disjoint union of copies of $BG$.

 \item flat site (fppf). This is OK, but not great. It is bad for
differential geometry.

 \item lisse-\'etale, which has its own problems, but it is the best
we can do.
\end{itemize}

\begin{definition}
  Let $\X$ be an algebraic stack over $S$. The \emph{lisse-\'etale
site} $\Liset(\X)$ of $\X$ has objects smooth morphisms $u:U\to \X$
with $U$ an algebraic space (notation: we will denote such an object
by $(U,u)$) and morphisms $(U,u)\to (V,v)$ are pairs $(f,f^\flat)$
where $f:U\to V$ is an $S$-morphism and $f^\flat:u\to v$ is an
(automatically cartesian) arrow in $\X$ over $f$. A collection
$\{(U_i,u_i)\xrightarrow{(f_i,f_i^\flat)} (V,v)\}$ is a covering if
$\coprod U_i\to V$ is \'etale surjective. Let $\X_\liset $ be the
associated topos. We define the structure sheaf $\O_{\X_\liset }$ to
be $(U,u)\mapsto \Ga(U_{et},\O_{U_{et}})$.
\end{definition}
\begin{remark}
  Concretely, to give a sheaf $F$ in this site is equivalent to a
sheaf $F_{(U,u)}$ on the \'etale site $U_{et}$ for every object
$(U,u)\in \Liset(\X)$ and transition maps (don't have to be
isomorphisms) $\phi_f:f^{-1}F_{(V,v)}\to F_{(U,u)}$ for every
morphism $(f,f^\flat):(U,u)\to (V,v)$ compatible with composition
($\phi_{gf}=\phi_f \circ f^{-1}\phi_{g}$) \anton{I think we have to
require that when $f$ is \'etale, $\phi_f$ is an isomorphism. This is
to verify the sheaf condition on the lisse-\'etale sheaf you build
from such data}.

  \anton{write out the verification better} Given a sheaf $F\in
\X_\liset$, we define $F_{(U,u)}$ to be the restriction of $F$ to the
small \'etale site of $U$. It is easy to see that the sheaf condition
on $F$ implies the sheaf condition on $F_{(U,u)}$. For a morphism
$f^\flat:u\to v$ (which projects to $f:U\to V$), we get a morphism
$f^{-1}F_{(V,v)}\to F_{(U,u)}$ by the universal property of limits
(you have to remember the definition of $f^{-1}$); since this is
defined by a universal property, it probably automatically plays well
with composition. Note that if $f$ is \'etale, then this map is an
isomorphism (because the directed system over which you take a limit
has a terminal object).

  Conversely, if you have the data, define $F(U,u):=F_{(U,u)}(\id_U)$
and observe that the $\phi_f$ give you restriction maps. The sheaf
axiom follows from the fact that $\phi_f$ is an isomorphism when $f$
is \'etale. \anton{complete this}
\end{remark}
\begin{remark}
  If $\X$ is an algebraic space, then this is as close to the \'etale
topology as you can get: the inclusion $Et(\X)\hookrightarrow
\Liset(\X)$ is continuous and $Et(\X)$ has finite projective limits,
so we get a morphism of topoi $\e:\X_\liset \to \X_{et}$. Note that
$\e_*$ is exact: $\e_*F=F_{(\X,\id_\X)}$ is exact because do
something on each \'etale something \anton{}.
\end{remark}

It might be true that the Lisse-Lisse site gives you the same topos
(even though it looks like it should give a smaller topos) because
every smooth morphism \'etale locally has a section. I dunno.

\begin{example}
  $\Liset(\M_{1,1})$ has objects pairs $(U,E_U)$ where $U$ is an
algebraic space and $E_U$ is an elliptic curve over $U$ such that
$E_U:U\to \M_{1,1}$ is smooth \anton{how should I think about this
condition?}. Morphisms incorporate the automorphisms: a morphism
$(U,E_U)\to (V,E_V)$ is a cartesian diagram
%   \[\xymatrix{
%     E_U\ar[r]^{f^\flat}\ar[d] & E_V\ar[d]\\
%     U\ar[r]^f & V
%   }\]
  \[\xymatrix{
   E_U\ar[d]\ar[r]^{f^\flat} & E_V\ar[d]\\
   U\ar[r]^f \ar@/^/[u]^{e_U} & V \ar@/_/[u]_{e_V}
  }\]
  You have choices for $f^\flat$.

  What is an example of a sheaf? If $(U,E_U)$ is an object, define
$\w_U=e^*\Om^1_{E_U/U}$ ($e$ is the section $U\to E_U$), which is a
locally free sheaf of rank 1 on $U_{et}$. We have a cartesian diagram
as above. Then we get $f^*e_V^*\Om^1_{E_V/V}\cong e_U^*\Om^1_{E_U/U}$
\anton{here upper $f^*$ means $f^{-1}$; note that
$\Om^1_{E_U/U}=f^{\flat*}\Om^1_{E_V/V}$} so we get a sheaf $\w\in
\M_{1,1,\liset}$. This would work even in the big fppf site.

  Note that this is an $\O_{\M_{1,1}}$-module.
\end{example}
\begin{definition}
  A sheaf $F$ of $\O_\X$-modules is \emph{quasi-coherent} if each
$F_{(U,u)}$ is quasi-coherent and for every morphism
$(f,f^\flat):(U,u)\to (V,v)$, the map $f^\star F_{(V,v)}\to
F_{(U,u)}$ is an isomorphism.
\end{definition}
If $F_{(V,v)}$ is a sheaf of ideals and $f$ is not flat, then
$f^*F_{(V,v)}$ need not be a sheaf of ideals \dots we'll take care of
this later.
\begin{remark}
  We can also talk about locally free sheaves of finite rank (each
$F_{(U,u)}$ should be locally free of finite rank). In the locally
noetherian case, we can talk about coherent sheaves. If $\X$ is an
algebraic space, a sheaf $F$ of $\O_{\X_\liset }$-modules is
quasi-coherent if and only if $F_X=\e_*F$ is quasi-coherent and the
adjunction map $\e^*F_\X=\e^*\e_*F\to F$ is an isomorphism.
\anton{this gives an equivalence of categories of quasi-coherent
sheaves?}
\end{remark}
\begin{example}
  Let $k$ be a field, and let $G$ be a smooth group scheme over $k$.
Then $\qco(BG)\cong Rep_k(G)$.
  \[\xymatrix{
   F \spec k\ar[r]^{f_g}& \spec k, F \rlap{$k$-vector space}\ar[d]  \\
   & BG, \F \rlap{quasi-coherent sheaf}
  }\]
  for every $g\in G(k)$, you get a map $f_g$ (something about the
${}^\flat$s). Thus, we get an action of $G(k)$ on $F$. Exercise to
check that this actually defines an equivalence of categories.

  What is $H^0(BG,\F)$? $\spec k\to BG$ is a smooth surjection, so
$H^0(BG,\F)\hookrightarrow F$.
  \[\xymatrix{
    & \spec k\ar[d]\\
   T\ar[r]\ar[ur]^{et\ locally} & BG
  }\]
  It turns out that $H^0(BG,\F)\cong F^G$. This can be generalized to
$\X=[Y/G]$ over any base $S$ where quasi-coherent sheaves become
$G$-linearized and \dots
\end{example}
\underline{Functoriality}: If $f:\X\to \Y$ is a morphism of algebraic
stacks, we have no hope of getting a morphism of topoi \anton{we saw
this for schemes?}. We want adjoint functors $f_*:\X_\liset \to
\Y_\liset $ and $f^{-1}:\Y_\liset \to \X_\liset $. $f^{-1}$ is not
exact, so this is not a morphism of topoi.
 
If $f$ is representable, then we have a continuous map $\Liset(\Y)\to
\Liset(\X)$ given by $(U\xrightarrow u \Y)\mapsto (U\times_\Y \X\to
\X)$. To get the functors $f_*$ and $f^{-1}$, there are two options:

(1) note that in the definition of the lisse-\'etale site, we could
have taken morphisms from stacks instead of from algebraic spaces.
Define $\widetilde{\Liset}(\X)$ to have objects representable smooth
morphisms $\U\to \X$ and morphisms are
\[\xymatrix{
 \U\ar[rr]^f \ar[dr]^{}="a" & & \U'\ar[dl]_(.2){}="b"\\
 & \X
 \ar@{=>}^{f^\flat} "a";"b"
}\]
Then check that this gives you the same topos.

(2) First define $f^{-1}$ on representable sheaves. If $y:Y\to \Y$ is
smooth and $u:U\to \Y$, then
$h_{(Y,y)}(U,u)=\hom_{lis-et(\Y)}\bigl((U,u),(Y,y)\bigr)$. Then
$f^{-1}(h_{(Y,y)})(U\to \X)=\Bigl\{ \xymatrix{U\ar[d]_u \ar[r] &
Y\ar[d]^y_{}="b"\\ \X\ar[r]^{}="a" & \Y \ar@{=>} "a";"b"}\Bigr\}$.
Then define $(f_*F)(Y\xrightarrow y \Y) = \hom_{\Y_\liset
}(h_{(Y,y)},f_*F) = \hom_{\X_\liset
}(f^{-1}h_{(Y,y)},F)=\Ga(\X_\liset |_{f^{-1}(Y,y)},F)$

Recall that if $T$ is a topos and $G\in T$, then we can define $T/G$,
which turns out to be a topos. For $G\in \Y_\liset $, define $(\hat
f^{-1} G)(U\xrightarrow u \X)$ to be the limit over $I_{(U,u)}$, the
category of diagrams
 \[\xymatrix{U\ar[d]_u \ar[r] & V\ar[d]^v_{}="b"\\
 \X\ar[r]^{}="a" & \Y \ar@{=>} "a";"b"}
 \]
 of $G(V,v)$. Then $f^{-1}G$ is the sheaf associated to the presheaf
$\hat f^{-1}G$.
\begin{remark}
  $I_{(U,u)}$ is not filtering, which is what ruins the exactness of
$f^{-1}$.
\end{remark}
\begin{remark}
  There is a natural map $f^{-1}\O_{\Y_\liset }\to \O_{\X_\liset }$.
So we can define $f^*G=f^{-1}G\otimes_{f^{-1}\O_{\Y_\liset
}}\O_{\X_\liset }$ for an $\O_\Y$-module $G$.
\end{remark}
\begin{proposition}
  Let $G$ be a quasi-coherent sheaf on $\Y$. Then $f^*G$ is also
quasi-coherent.
\end{proposition}
\begin{proof}
  Note that $f^*G=(\hat f^{-1}G\otimes_{f^{-1}\O_\Y} \O_\X)^a$
because they have the same universal property \dots use the
adjunction. We need to compute $\lim_{I_{(U,u)}} (\tilde f^*
G_V)(U)$. Define $\bbar I_{(U,u)}$ to be the poset whose elements are
$(g:U\to V)\in \I_{(U,u)}$ where we say that $(g:U\to V)\ge (g':U\to
V')$ if there exists a $V\to V'$ in $I_{(U,u)}$.

  The category $I_{(U,u)}$ is almost filtering; we just squash
multiple arrows together.

  We have a surjection $I_{(U,u)}\to \bbar I_{(U,u)}$. The claim is
that
  \[\xymatrix{
   I_{(U,u)}\ar[dr]\ar[rr]^{\tilde f^*G_V} & & \Ga(U,\O_U)\mod\\
   & \bbar I_{(U,u)}\ar@{-->}[ur]
  }\]
\end{proof}
